%!TEX root = twig-gpu.tex

\section{Introduction}

Programming for hybrid architectures is a challenging task, in
large part due to the partitioned memory model they impose on
programmers. Unlike a basic SMP, devices must be set up and torn
down, processing synchronized, and data explicitly allocated on a
particular device and moved around within the memory hierarchy.
Programming systems such as CUDA\cite{cuda} and
OpenCL\cite{opencl} provide an interface for these operations, but
they are quite low-level. In particular, they do not distinguish
between the high-level computational and application logic of a
program, and the \emph{protocol logic} related to managing
heterogeneous devices. As a result, the different types of program
logic invariably become entangled, leading to excessively complex
software that is prohibitively difficult to develop, maintain, and
compose with other software. The problem we have described is
pervasive in programming for hybrid architectures; in this paper,
we will focus on the specific instance of this problem presented
by GPU-based accelerators.

We present a high-level programming language called \emph{Twig},
designed for expressing protocol logic and separating it from
computational and application logic. Twig also supports automated
reasoning about composite programs that can, in many cases, avoid
problems such as redundant memory copying. This allows Twig
programs often to retain the high performance of a lower-level
programming approach.

Crucially, Twig's role in the programming toolchain is to generate
code in a mainstream language, such as C. The generated code is
easily incorporated into the main program, which is then compiled
as usual. This minimizes the complexity that Twig adds to the
build process, and allows Twig code to interact easily with
existing code and libraries.

Twig achieves these goals by using data types to direct the
generation of code in the target language. In particular, we
augment existing data types in the target language with a notion
of \emph{location}, e.g., an array of floats located on a GPU, or
an integer located in main memory. In the following sections, we
first present related work, and then describe Twig's code
generation strategy and core semantics. Finally, we present an
example demonstrating the use of located types to generate code
for a GPU. In the example, we also show how Twig programs can be
automatically rewritten in order to minimize data movement.
