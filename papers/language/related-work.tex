%!TEX root = twig-language.tex

\section{Related Work}

There are many tools which incorporate some notion of typemaps which are in widespread use today. The idea originated in SWIG~\cite{swig}, a system for generating foreign function interfaces from C header files and some user-specified directives. Typemaps in Swig are robust, and do support user customization. However, the semantics of Swig's typemaps are ad-hoc and not especially flexible. Also, they are specialized to generate C code.

FIG~\cite{fig} introduced the notion of application-specific typemaps, and is quite similar to our own work in both its spirit and the actual semantics of its typemaps. Unlike Twig, however, FIG generates code for Moby~\cite{moby-classes}. Moby is, in some ways, a convenient target language -- its declarative structure and semantics are amenable to generation via System S~\cite{fisher00interop}. Indeed, FIG takes advantage of this fact by providing Moby-specific rules within FIG. Moby is not nearly as ubiquitous as C, however, and therefore not a very practical target language for many people.

There are many other tools that provide some kind of typemap language or facility in the service of their intended function, particularly foreign-function interface generators such as Charon~\cite{moby-interop-framework}, NLFFIGen~\cite{blume01}, etc. We feel that Twig could complement many of these sytems quite well, providing a foundational semantics for their typemap languages, while providing the abilty to generate C code.

Our abstract code generation model was based in part on our own previous work on a language called Wool~\cite{wool}.
