%!TEX root = twig-language.tex

\section{Implementation}

We wrote our implementation of Twig, called \texttt{twigc}, in
Haskell. The \texttt{twigc} expects two input files. The first
file contains a list of named expressions, including a
\texttt{main} expression, as described in
Section~\ref{sec:expressions}. The second file contains a ground
term (representing a C type), used as input to the main
expression.

Our version of \texttt{twigc} must be configured with a mapping
from terms to C types. The user provides this mapping with a
simple key/value text file.

If the input value can be successfully rewritten using the main
rule expression provided, then Twig will output the rewritten term
along with the generated block of C code. This code block may be
redirected to a separate file and included in a C program using
the \texttt{\#include} directive.

We are examining ways in which this process might be more easily
incorporated into a typical C programmer's workflow.

\subsection{Code Generation}

Our implementation supports the generation of C code, and adds
some extra features to support that language. This includes such
details as managing type declarations, support for parameterized
blocks, and for ``closing'' blocks, which are generated as
variables go out of scope and are intended to be used to free
resources. We are working on incorporating these features into the
abstract model.
