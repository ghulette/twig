%!TEX root = twig-language.tex

\subsection{Expressions}
\label{sec:expressions}

Twig \emph{expressions} can be either \emph{primitive rules}
(Section~\ref{sec:semantics:prims}) or else built from other
expressions using operators (Section~\ref{sec:semantics:ops}). An
expression $s$ maps terms $T$ to elements of the set $(T \times M)
\cup \{\bot\}$, i.e., either a pair $(t',m)$ where $t' \in T$ and
$m$ is a \emph{block} of generated code in the set $M$ (see
Section~\ref{sec:code-gen}), or else the special, distinguished
value $\bot$. Formally, $s$ is a function:

\[
s : T \to ((T \times M) \cup \{\bot\})
\]

Following Fig's notation, we use $\bot$ to denote ``failure.'' In
particular, $\bot$ is used in the semantics for the operators
described in Section~\ref{sec:semantics:ops}.

As with System S and Fig, Twig allows expressions to be named. An
expression's name may be used in place of itself within other
expressions. The syntax is

\[
v \mtt{ = } e
\]

where $v$ is a valid expression name and $e$ is an expression, as
described below. A Twig program is a list of such name/expression
assignments. There is a special expression name, \texttt{main},
which designates the top-level expression for the program.
