%!TEX root = twig-language.tex

\subsection{Operators}
\label{sec:semantics:ops}

Expressions can be combined using Twig's operators. In the
following semantics, let $t$ range over terms, $m$ range over
blocks, and $s$ range over expressions, i.e., either a primitive
rule, or else another expression built with operators.

The \emph{sequence} operator, written as an infix semi-colon
(\texttt{;}), chains the application of two rules together by
sending the output of the first to the input of the second. The
combined expression fails if either sub-expression fails (see
Figure~\ref{fig:seq}). With this operator, simple rules can be
composed into multi-step transformations. Upon success, the result
blocks are combined sequentially using the block sequence
operation (see Section~\ref{sec:code-gen:seq}).

\begin{figure}[ht]
\[
\infer
  {t \arr{s_1;s_2} (t'',m_1+m_2)}
  {t \arr{s_1} (t',m_1) \quad t' \arr{s_2} (t'',m_2)}
\]
\[
\infer
  {t \arr{s_1;s_2} \bot}
  {t \arr{s_1} \bot}
\qquad
\infer
  {t \arr{s_1;s_2} \bot}
  {t \arr{s_1} (t',m) \quad t' \arr{s_2} \bot}
\]
\caption{Semantics for sequence operator}
\label{fig:seq}
\end{figure}

\emph{Left-biased choice}, written as a vertical bar (\texttt{|}),
will attempt to apply the first rule expression to the input, and
if it succeeds then its output is the result (see
Figure~\ref{fig:choice}). If it fails, it attempts to apply
the second rule instead. This operator allows different code to be
generated depending on the input type.

\begin{figure}[ht]
\[
\infer
  {t \arr{s_1|s_2} (t',m_1)}
  {t \arr{s_1} (t',m_1)}
\qquad
\infer
  {t \arr{s_1|s_2} (t',m_2)}
  {t \arr{s_1} \bot \quad t \arr{s_2} (t',m_2)}
\]
\[
\infer
  {t \arr{s_1|s_2} \bot}
  {t \arr{s_1} \bot \quad t \arr{s_2} \bot}
\]
\caption{Semantics for left-biased choice}
\label{fig:choice}
\end{figure}

Figure~\ref{fig:basic1} and Figure~\ref{fig:basic2} give the
semantics for Twig's other basic operators. Identity (\texttt{T})
will always succeed, returning its input and an identity block,
failure (\texttt{F}) will always return \texttt{$\bot$}. Test
(\texttt{?}) takes a single expression as a parameter and succeeds
only if its argument succeeds, returning the original term.
Negation \texttt{$\lnot$} also takes a single expression argument,
and succeeds only if its argument fails, returning the original
term.

\begin{figure}[ht]
\[
\infer
  {t \arr{\mathtt{T}} (t,I)}
  {}
\qquad
\infer
  {t \arr{\mathtt{F}} \bot}
  {}
\]
\caption{Semantics for Success (\texttt{T}) and Failure (\texttt{F})}
\label{fig:basic1}
\end{figure}

\begin{figure}[ht]
\[
\infer
  {t \arr{?s} (t,I)}
  {t \arr{s} (t',m)}
\qquad 
\infer
  {t \arr{?s} \bot}
  {t \arr{s} \bot}
\]
\[
\infer
  {t \arr{\lnot s} \bot}
  {t \arr{s} (t',m)}
\qquad 
\infer
  {t \arr{\lnot s} (t,I)}
  {t \arr{s} \bot}
\]
\caption{Semantics for Test (\texttt{?}) and Negation ($\lnot$)}
\label{fig:basic2}
\end{figure}

Twig also provides some operators especially for tuples. For each
of the following operators, there are some additional semantics
related to tuples that we have elided for lack of space.
Informally, the omitted semantics state that a tuple operator will
fail (i.e., return $\bot$) if the input is not a tuple, or if the
expression references an element that is outside the tuple bounds.

% \begin{figure}[ht]
% \[
% \infer
%   {f(\ldots) \arr{s} \bot}
%   {f \neq \mathtt{tuple}}
% \qquad
% \infer
%   {\mathtt{tuple}(t_1,\ldots,t_n) \arr{s(i)} \bot}
%   {i > n}
% \]
% \caption{Common semantics for tuple operators}
% \label{fig:all-tuples}
% \end{figure}

The \emph{congruence} operator applies a tuple of expressions to
the elements of a tuple term, pairwise, and returns a tuple of
results. It fails in case any of the individual rule applications
fail. Upon success, the result block is the parallel composition
(see Section~\ref{sec:code-gen:par}) of the individual result
blocks. The semantics for congruence are shown in
Figure~\ref{fig:congruence}.

\begin{figure}[ht]
\[
\infer
  {\mathtt{tuple}(t_1,\ldots,t_n)\arr{(s_1,\ldots,s_n)} (\mathtt{tuple}(t_1',\ldots,t_n'),m_1 \times \ldots \times m_n)}
  {t_1 \arr{s_1} (t_1',m_1) \quad \ldots \quad t_n \arr{s_n} (t_n',m_n)}
\]
\[
\infer
  {\mathtt{tuple}(\ldots,t_i,\ldots)\arr{(\ldots,s_i,\ldots)}\bot}
  {t_i \arr{s_i} \bot}
\]
\caption{Semantics for congruence operator}
\label{fig:congruence}
\end{figure}

The family of \emph{branch} operators apply a single rule to one,
all, or some of the tuple's elements, depending on the variant.
Figure~\ref{fig:branch} shows the semantics for the \texttt{\#one}
branch operator, which applies an expression $s$ to a single
element: the first element, from left to right, for which $s$ does
not fail. The other elements of the tuple are unchanged. The
branch operation fails if $s$ fails for all the elements. The
branch operator constructs the result block using a fanout block,
(see Section~\ref{sec:code-gen:special}). We elide details for the
other branch operators for lack of space.

\begin{figure*}[ht]
\[
\infer
  {\mathtt{tuple}(\ldots,t_i,\ldots) \arr{\mathtt{\#one}(s)} (\mathtt{tuple}(\ldots,t_i',\ldots),(I \times \ldots \times m_i \times \ldots \times I))}
  {t_i \arr{s} (t_i',m_i)}
\]
\[
\infer
  {\mathtt{tuple}(t_1,\ldots,t_n) \arr{\mathtt{\#one}(s)} \bot}
  {t_1 \arr{s} \bot \quad \ldots \quad t_n \arr{s} \bot}
\]
\caption{Semantics for branch-one operator}
\label{fig:branch}
\end{figure*}

The \emph{projection} operator extracts a single indexed element
from a tuple (see Figure~\ref{fig:projection}). Similarly, the
\emph{path} operator applies a rule to a single indexed tuple
element, leaving the other elements unchanged (see
Figure~\ref{fig:path}).

\begin{figure}[ht]
\[
\infer
  {\mathtt{tuple}(\ldots,t_i,\ldots) \arr{\#i} (t_i,\Pi(i))}
  {}
\]
\caption{Semantics for projection operator}
\label{fig:projection}
\end{figure}

\begin{figure}[ht]
\[
\infer
  {\mathtt{tuple}(\ldots,t_i,\ldots) \arr{\#i(s)} (\mathtt{tuple}(\ldots,t_i',\ldots), I \times \ldots \times m_i \times \ldots \times I)}
  {t_i \arr{s} (t_i',m_i)}
\]
\[
\infer
  {\mathtt{tuple}(\ldots,t_i,\ldots) \arr{\#i(s)} \bot}
  {t_i \arr{s} \bot}
\]
\caption{Semantics for path operator}
\label{fig:path}
\end{figure}

Finally, the \emph{permutation} operator allows arbitrary
permutation of a tuple's elements, including duplicating or
dropping elements. The semantics are given in
Figure~\ref{fig:permute}. We treat $\mbox{permute}_1$, where the
permutation expects a single input, specially -- in this case, we
treat non-tuple input terms as tuples of length one. This allows
the permutation operator to be used to replicate single terms. The
semantics for this special case are given in
Figure~\ref{fig:permute-one}.

As a convenience, we also provide a ``fan out'' operator which is
defined as a $\mbox{permute}_1$ followed by a congruence. The
definition is given in Figure~\ref{fig:fan}.

\begin{figure*}[htb]
\[
\infer
  {\mathtt{tuple}(t_1,\ldots,t_n) \arr{\mathtt{\#permute}_n(x_1,\ldots,x_m)} (\mathtt{tuple}(t_{x_1},\ldots,t_{x_m}),\Pi_w(y_{x_1},\ldots,y_{x_m}))}
  {}
\]
\begin{eqnarray*}
w   &=& \sum_{j=1}^n \mbox{width}(t_i)\\
b_i &=& \left\{
  \begin{array}{cl}
    0 & \mbox{if } i = 1\\
    \sum_{j=1}^{i-1} \mbox{width}(t_i) & \mbox{if } i > 1
  \end{array}
\right.\\
y_i &=& b_i+1,\ldots,b_i + \mbox{width}(t_i)
\end{eqnarray*}
\caption{Semantics for permutation operator}
\label{fig:permute}
\end{figure*}


\begin{figure}[htb]
\[
\infer
  {t \arr{\mathtt{\#permute}_1(1,\overset{n}{\ldots},1)} (\mathtt{tuple}(t,\overset{n}{\ldots},t),\Pi_1(y,\overset{n}{\ldots},y))}
  {}
\]
\begin{eqnarray*}
y &=& 1,\ldots,\mbox{width}(t)
\end{eqnarray*}
\caption{Semantics for permute 1 operator}
\label{fig:permute-one}
\end{figure}


\begin{figure}[ht]
\[
\mathtt{\#fan}(n) \equiv \mathtt{\#permute}_1(\underbrace{1,\ldots,1}_\text{n})
\]
\caption{Semantics for fan operator}
\label{fig:fan}
\end{figure}

Twig also includes a fixed-point operator, allowing it to express rules for handling recursively defined data types. We have omitted the semantics for lack of space.
