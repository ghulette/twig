%!TEX root = twig-language.tex

\subsection{Future work}

One of Twig's limitations is in its handling of \emph{container} types. Arrays are the most common example of a container, but they include any type which may contain elements of other types, such as trees or lists. In order to properly handle container types, Twig needs to be extended with higher-order rules, i.e., rules which take other rules as paramters. We are currently working on a syntax and semantics for rules such as this:

\begin{verbatim}
[array(X) -> array(Y) | X -> Y]
\end{verbatim}

This should be read as a rule which transforms an array of any type $X$ to an array of another type $Y$, given a rule to transform a single element of type $X$ to type $Y$.

Other improvements to Twig may include extending our implementation to support other target languages, such as Java or Python, which would take advantage of our abstract code generation model.

\section{Conclusion}

We have presented Twig, a typemapping language that may serve as a more flexible alternative to typemapping facilities commonly found in tools such as Swig. Typemaps in Twig may be composed in a number of useful ways, and include useful features like tuples, polymorphic rules, and runtime choice based on the input type.

Twig also incorporates a flexible model for code generation. While our current implementation is focused on generating C, other procedural languages, such as Java or Python, may easily to generated instead.
