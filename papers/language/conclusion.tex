%!TEX root = twig-language.tex

\section{Future work}

Twig is limited in its handling of \emph{container} types, such as
arrays, lists, or trees. In order to properly handle this kind of
data, Twig needs to be extended with higher-order primitive rules,
i.e., rules which take expressions as parameters. We are currently
working on a syntax and formal semantics for rules such as:

\begin{verbatim}
[array(X) -> array(Y) | X -> Y] <<< ... >>>
\end{verbatim}

This should be read as a rule which transforms an array of any
type $X$ to an array of another type $Y$, given a rule to
transform a single element of type $X$ to type $Y$.

Other improvements to Twig may include extending \texttt{twigc} to
support other target languages, such as Java or Python. This would
take advantage of our abstract code generation model.

\section{Conclusion}

We have presented Twig, a language for typemaps that may serve as
a more flexible alternative to the languages found in tools such
as SWIG. We have demonstrated how Twig typemaps are created and
composed, and shown how our language incorporates useful features
such as tuples, polymorphic rules with variables, and runtime
choice based on the input type.

Twig includes a flexible model for code generation. While our
current implementation is focused on generating C, other
languages, such as Java or Python, could be generated instead.
