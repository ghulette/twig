%!TEX root = twig-language.tex

\section{Introduction}

Twig is a new language for writing \emph{typemaps} -- programs that transform data from one type to another, while preserving (as much as possible) the underlying value of the data. Typemaps have proven useful in many kinds of programming and especially automated code generation, where we require a transformation to pass a single nominal value across a pair of mismatched types that we know to be interchangeable in some way. The best-known example of this problem is found in multi-language programming. For example, a programmer may wish to pass a Python integer to a C function, where a C \texttt{int} is expected. If we have a typemap that performs the transformation from Python integers to C integers, then an automated tool can generate the conversion code automatically, and expose the C function in Python via a generated wrapper.

There are a number of existing tools and languages for creating typemaps and generating code from them. Twig builds on existing typemap tools in several ways.

First, Twig's typemaps are composable, i.e., new typemaps may be constructed by combining old ones. Thus, complex typemap transformations may be built from simpler ones. Our notion of typemap composition is based on the formalisms used in Fig\cite{fig} and System S\cite{system-s}, but we extend and refine that work in some key ways.

Second, Twig incorporates a robust, formal model of code generation. This allows Twig to generate code based on typemaps in many different target languages.

Finally, Twig includes a facility for \emph{reducing} typemaps by exploiting identity relationships among typemap expressions. Some reductions are based on a universally-applicable algebra of typemaps, while others are domain-specific and must be described by the user. We have shown in prior work that typemap reduction can be used to optimize certain transformations. Reductions are covered in our previous work, and we will not address them further here.

In this paper, we will describe Twig's formal language structure, and then show how this structure allows us to express complex typemaps more concisely than with traditional tools. First, we review existing approaches to typemaps and related problems. Second, we present the semantics for Twig's code generation model, and then the semantics for the typemap language itself. Third, we present a typemap example in Swig, and show how the same problem can be solved more concisely and clearly in Twig. Finally, we conclude with ideas for future work.
