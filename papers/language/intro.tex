%!TEX root = twig-language.tex

\section{Introduction}

Introduce Twig as a more formal alternative to Swig.

Swig introduced the concept of \emph{typemaps}, i.e. snippets of code that could
be used to convert data from one type to another in a given language. While
powerful, typemaps are not especially flexible -- you need to have a typemap for
every pair of types you want to convert. Creating such a set of typemaps is an
$n^2$ problem given $n$ types. Once we consider user-defined types, the problem
may become quite large.

Twig builds on the ideas introduced in Swig and Fig.

extends the ideas of Swig, with a notion of composable typemaps. We base
our design on Fig, but extend it in a few important ways.

First, Twig is not restricted to generate Moby bindings. It uses a novel model
for code generation that is both abstract and formal, and which is easy to
implement for both imperative and functional-style languages.

Second, Twig adds support for \emph{functors} -- a notion of typemaps
parameterized by other typemaps. Functors allow us to compose complex, layered
type conversions.

Third, Twig adds \emph{reductions} -- the ability for authors to provide rules
which can rewrite composite typemaps to simpler transformations according to
either general typemap algebra or domain-specific rules.
