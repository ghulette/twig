%!TEX root = twig-language.tex

\section{Evaluation}

Twig has several advantages over typemap facilities such as those found in Swig.

\begin{enumerate}

\item Sequencing: simple typemaps can be composed in sequence to produce more complex transformations.

\item Choice: With choice combinator, a single typemap expression may be used to generate multiple variations of a transformation, depending on the input type.

\item Tuples: Twig allows sets of types to be mapped together, using tuples. This is a common problem -- consider function argument lists, or a pointer paired with a length to form an array.

\item Type variables: allows polymorphic typemaps, e.g., \texttt{[ptr(A) -> A]}.
  
\item Target langauge flexibility: Twig can be extended to generate target languages other than C.

\end{enumerate}

Twig also improves on Fig in a number of ways:

\begin{enumerate}

\item Type variables: allows polymorphic typemaps.

\item Target language: Fig is intimately tied to its target language, Moby; whereas Twig can be extended to generate a variety of mainstream languages, notably C.

\end{enumerate}

We will now walk through the construction of a simple typemap for converting from polar to Cartesian coordinates. First, we define a typemap that takes a pair representing the distance and angle of a point in polar coordinates to the $x$ component of the equivalent point in Cartesian space.

\begin{verbatim}
p2cxf = [(float,float) -> float] <<<
  $out = $in1 * cosf($in2);
>>>
\end{verbatim}

We name this rule \texttt{p2cxf}, and associate with it a block of C code which performs the transformation. We can construct a similar rule which converts the same values to the $y$ component in Cartesian space.

\begin{verbatim}
p2cyf = [(float,float) -> float] <<<
  $out = $in1 * sinf($in2);
>>>
\end{verbatim}

Now, with these two rules we can use the \emph{fan out} operator to take a pair of \texttt{float}s, and duplicate it. Using the sequence operator, we can pass the result to the congruence of the two rules. This effectively defines a new rule, which will convert a pair of \texttt{floats} representing polar coorindates to a pair of floats in Cartesian space.

\begin{verbatim}
p2cf = #fan(2);{p2cxf,p2cyf}
\end{verbatim}

We name this rule \texttt{p2cf}. We can also define a set of  rules which are essentially the same, except that they transform on \texttt{double} values instead of \texttt{float}s (and which uses the appropriate C function variants).

\begin{verbatim}
p2cxd = [(double,double) -> double] <<<
  $out = $in1 * cos($in2);
>>>

p2cyd = [(double,double) -> double] <<<
  $out = $in1 * sin($in2);
>>>

p2cd = #fan(2);{p2cxd,p2cyd}
\end{verbatim}

Now we can combine these two rules, using the \emph{choice} operator, to create a typemap that will work on pairs of either  \texttt{double}s or \texttt{float}s.

\begin{verbatim}
p2c = p2cdf | p2cd
\end{verbatim}

Why would this be useful? Say we are writing an application that uses two different libraries, with two different \texttt{struct} representations for polar coordinates. One has \texttt{double} fields, the other \texttt{float}, like so:

\begin{verbatim}
struct PolarF {
  float r;
  float theta;
};

struct PolarD {
  double dist;
  double angle;
};
\end{verbatim}

and we want to be able to convert those types to a single type representing a Cartesian point with \texttt{float} fields, like so:

\begin{verbatim}
struct Pt {
  float x;
  float y;
};
\end{verbatim}

We can extract the fields of both polar \texttt{structs} to their respective tuple types (note that we must also inform Twig that the terms \texttt{polard}, \texttt{polarf}, \texttt{pt} correspond to the appropriate \texttt{struct} in C).

\begin{verbatim}
pexf = [polarf -> (float,float)] <<<
  $out1 = $in.r;
  $out2 = $in.theta;
>>>

pexd = [polard -> (double,double)] <<<
  $out1 = $in.dist;
  $out2 = $in.angle;
>>>

pex = pexf | pexd
\end{verbatim}

The rule \texttt{pex} will extract the fields, with the appropriate type, from either \texttt{struct} type. We can then sequence this rule with \texttt{p2c}, defined above, and convert either structure to Cartesian coordinates.

To put the resulting pair of either \texttt{double}s or \texttt{float}s into the \texttt{Pt} structure, we need just a few more rules. The rule \texttt{d2f} will just cast a \texttt{double} to a \texttt{float}, while \texttt{mkpt} will construct a \texttt{Pt} \texttt{struct} from a pair of \texttt{float}s.

\begin{verbatim}
d2f = [double -> float] <<<
  $out = (float)$in;
>>>
  
mkpt = [(float,float) -> pt] <<<
  $out.x = $in1;
  $out.y = $in2;
>>>
\end{verbatim}

The following rule expression will convert a pair of \texttt{double}s to a pair of \texttt{float}s if needed, or simply pass through a pair of \texttt{float}s, and then put the result into a \texttt{Pt} struct.

\begin{verbatim}
({d2f,d2f}|T);mkpt
\end{verbatim}

The full expression, which will convert either a \texttt{polard} or \texttt{polarf} to \texttt{pt} is as follows.

\begin{verbatim}
polarToPt = pex;p2c;({d2f,d2f}|T);mkpt
\end{verbatim}

We can invoke Twig with this typemap as its program. To generate the C code to perform the transformation, we apply \texttt{polarToPt} to one of the terms \texttt{polarf} or \texttt{polard}. If we choose \texttt{polarf}, Twig will generate the code to convert a \texttt{PolarF} \texttt{struct} to \texttt{Pt} \texttt{struct}, like so:

\begin{verbatim}
struct Pt polarToPt(struct PolarF gen1) {
  float gen2,gen3,gen4,gen5;
  struct Pt gen6;
  gen2 = gen1.r;
  gen3 = gen1.theta;
  gen4 = gen2 * cosf(gen3);
  gen5 = gen2 * sinf(gen3);
  gen6.x = gen4;
  gen6.y = gen5;
  return gen6;
}
\end{verbatim}

The advantage of the typemapping approach is that we are able to capture the process of transforming a polar coordinate and separate it from the types we happen to be using. Also, the resulting typemap is polymorphic, in that we may use a number of types as input. The output type could have been dependent on the input type as well, although in this example it was not.

To accomplish the same task in Swig is not stricly possible, since Swig typemaps are used exclusively across languages. This makes sense in Swig, since typemaps are monolithic.

