%!TEX root = twig-language.tex

\subsection{Primitive Rules}
\label{sec:semantics:prims}

The simplest Twig expressions are \emph{primitive rules}, which
describe a single transformation step. Since Twig terms represent
types in a target language, a primitive rule in Twig describes how
to transform an instance of one type into an instance of another
in that language.

The syntax for primitive rules is

\[
\mtt{[} p_1 \mtt{ -> } p_2 \mtt{]} \mtt{<<< } m \mtt{ >>>}
\]

where $p_1$ is the \emph{input pattern}, $p_2$ the \emph{output
pattern}, and $m$ is a block of code (see
Section~\ref{sec:code-gen}). Input and output patterns are terms,
but which can also contain \emph{variables}.

Informally, a primitive rule transforms $t$ to $t'$ with an
associated block $m$ if and only if the application of rule to
value $t$ succeeds. In this case we write $t \arr{s} (t',m)$.
Otherwise, if the application of $s$ to $t$ fails, e.g., if $t$
does not match the input pattern in $s$, then we write $t \arr{s}
\bot$. In this case, the block is not produced.

For example, in C it is easy to convert an integer value to
floating point. Twig's syntax for writing this rule is as follows:

\begin{verbatim}
[int -> float] <<< $out = (float)$in; >>>
\end{verbatim}

In this example, if the input term to the expression
\emph{matches} the term \texttt{int}, then the output of the
expression will be the term \texttt{float} along with a code block
(the text between \texttt{<<<} and \texttt{>>>}, see below). If
the input term does not match \texttt{int} then the output will be
$\bot$.

Input and output patterns can have \emph{variables} in place of
terms or sub-terms. For example the rule

\begin{verbatim}
[ptr(X) -> X] <<< $out = &$in; >>>
\end{verbatim}

describes a transformation of any C pointer type to its referent.
The variable \texttt{X} is bound to the corresponding value of the
matched input on the right, and that value is substituted for the
variable where it appears on the left. Variables may stand in
place of a single term only, not constructors; e.g., patterns such
as \texttt{[X(int)~->~X]} are not allowed.

When an input term is \emph{matched}, we mean that it is
unified~\cite{baader98rewriting} with the input pattern. In fact,
Twig's matching algorithm is simpler than full unification, since
there is no equational theory and the input term may not contain
variables (i.e., must be a ground term). If unification is
successful, we say that the term \emph{matched}. The bound
variables (if any) are saved in order to construct the output term
via substitution. We omit further formalization of matching here
for lack of space; see \cite{baader98rewriting,system-s} for
details.

% Probably should talk about variable binding, e.g. how (X,X) will match a pair of identical terms.

\subsubsection{Blocks}

Primitive rules are associated with have a block. In Twig's syntax, blocks are surrounded by triple-angle braces and appear immediately after a primitive rule, like so:

\begin{verbatim}
[int -> float] <<<
  $out = (float)$in;
>>>
\end{verbatim}

The contents of the block depends on the target language; in this case, we have provided a block of escaped C code (as described in Section~\ref{sec:code-gen:c}). The target language must be specified as a parameter to the Twig runtime (in our current implementation, as a command line option). This means that the text between the \verb|<<<| and \verb|>>>| is interpreted based on the chosen block implementation. A target language specific procedure is passed the text, along with the input and output term(s). It can then construct the block however it sees fit. In our C block implementation, for example, variable names are generated and the terms are used to determine the appropriate types for declaring those variables in a prelude section of the code. Other target languages might choose very different block construction mechanisms.

It is important to understand that Twig's semantics do not require that block contents be checked in any way (although an implementation could do this), and that the code generation procedure is strictly syntactic. This scheme is similar to that used by SWIG~\cite{swig}.
