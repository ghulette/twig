\section{Related Work}

Kennedy's \emph{telescoping languages} work sought to provide a
similar ability to build high-level problem-solving languages that
used scripting languages to coordinate functionality present in
domain-specific libraries~\cite{kennedy00telescoping}.  This work took
a different approach to the problem, focusing on compiler optimization
methods and targeting potentially distributed, grid-based
environments.  Interestingly, one of the compilation techniques that
was called out as part of the telescoping languages strategy was that
of automated recognition and exploitation of identities.  This would
allow a compilation tool to recognize instances when compositions of
functions could be replaced with more efficient equivalent
implementations.  Our work adopts a similar strategy in identifying
compositions of rules that are inverses of each other and can
therefore be eliminated.

Domain-specific language approaches have had notable success in the
computational science field, an exemplar being the Tensor Contraction
Engine (TCE)~\cite{baumgartner05synthesis}.  The TCE allows
computational chemists to write tensor contraction operations in a
high level language similar to Mathematica, leaving it up to the TCE
tool to generate the corresponding collections of loops that implement
the operations.  The advantage of this approach is that tedious and
often error-prone nested loops over many large arrays with a
correspondingly large number of indices can be both machine generated
and optimized.  Optimizations such as loop fusion, memory locality
management, and data distribution and partitioning in a parallel
machine can all be automated, versus previous approaches that required
very labor intensive hand written code.  The TCE is not a general
tool, and is only of use to programmers working with similar
tensor-based computations.

The closest work to that which we describe is Reppy's Application
Specific Foreign Function Interface Generator, FIG~\cite{reppy06fig}.
In that case, a similar term rewriting approach was taken specifically
in relation to the generation of bindings between programs in two
different programming languages.  Our work builds upon that of Reppy
and Song, and aims to provide a more generalizable method that focuses
solely on type-level mappings.  These mappings subsume the FFI
generation case (in which mappings are between two different type
systems), and other interesting applications like mapping between two
different libraries written in the same language.  This second case is
of particular interest when dealing with the problem of composing
complex software from smaller program units, where developers working
independently may have chosen different representations for data types
that are semantically identical.

With respect to the GPU programming context specifically, a number of language
approaches have been investigated to alleviate programmers of the tedious burden
of programming in languages like CUDA or OpenCL.  For example, ...


Discuss PyCUDA and OpenMPC. What about Jacket and other GPU-type interfaces
for languages like Matlab?

HPF was interesting because it explored extending the type system of Fortran
90 to add data distribution information, which feels similar to the Twig GPU
idea. definitely need to talk about that at some point in the paper.

Other general-purpose DSLs? Yampa/Arrows?

Accelerate GPU language in Haskell is similar.  
