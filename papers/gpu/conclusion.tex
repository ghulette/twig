%!TEX root = twig-gpu.tex

\section{Conclusion}

We have introduced the concept of separating the protocol logic that is inherent
to hybrid systems from the computational logic that forms the domain specific
intent of a program that uses the system. We have demonstrated that a type-based
approach can enforce this separation by making explicit in data types
information related to both the locale in which data resides, and the
representation of the data itself. By doing so, we allow the protocol logic of a
program to be expressed via operations exclusively on located types. Many
explicit programming chores become implicit features of the generated code, such
as declaring intermediate values or reducing redundant memory movement. Finally,
by adopting a code generation approach, we show that users of these higher level
abstractions are not prohibited from both tuning the resultant code and
composing together independently developed programs that utilize standardized
hybrid programming libraries like OpenCL or CUDA.

\subsection{Future work}

This paper focused on the formalism upon which we have built our type-based
system for encoding the protocol logic of hybrid system programming. Much work
remains to make this technique transparent to the user. First, we must develop
method for annotating existing code in order to enrich the type system to carry
located type metadata. Second, we must develop tools for extracting this
information and generating the Twig term representation that forms the input to
the rewriting and reduction techniques described here.

% Both cases can be
% implemented and demonstrated within extensible compiler frameworks such as the
% ROSE compiler infrastructure from Lawrence Livermore National Laboratory. Work
% to provide this capability via ROSE is currently underway as part of the DOE
% COMPOSE-HPC project.
