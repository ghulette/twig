%!TEX root = twig-gpu.tex

\section{Twig}

% Overview of how Twig works, basic semantics, code generation, simple
% example. Focus on importance of generating C - integration with existing
% methods, type-directed generation.

Twig is based a core semantics called System S~\cite{Visser:1998p333}. System
S was originally designed for specifying term rewriting systems (described
briefly in \ref{section:term-rewriting}). In Twig, we use the operators of
System S to combine primitive \emph{rules} into more complex transformations
on types. These transformations are then applied to a given type in order to
generate code. In this section we describe Twig's language.

\subsection{Rules}

The basic building blocks of a Twig program are called \emph{rules}. A rule simply specifies that a given type can be transformed into another type, and provides a snippet of code which performs the transformation. The snippets are essentially strings

\subsection{Values}

Values in Twig are types in the target language.

\subsection{Combinators}

Rules can be combined into more complex expressions using a fixed set of
operators.

\subsection{Code generation}

Our implementation of Twig generates C code, but in principle almost any
language could be generated instead.

\subsection{Reductions}

Reductions are a way to transform Twig expressions. Reductions may exploit
some application or domain knowledge about the nature of the rules, and as
such are usually developed alongside a set of rules.

% Explain reductions and how they can be used in domain- or
% application-specific ways.

\subsection{Term rewriting}
\label{section:term-rewriting}

% Brief overview of term rewriting and how it is used in reductions.

\subsection{Implementation}

% Talk briefly about how Twig is implemented, and how to use it to generate
% code.
