%!TEX root = twig-gpu.tex

\section{Introduction}

Programming for hybrid computers can be a challenging task, in large part due to
the partitioned memory model that they impose on programmers. Unlike a basic
SMP, data must be explicitly allocated on different devices and moved within the
memory hierarchy of a single compute node such that the appropriate processing
device can access it. This induces a considerable amount of protocol when
adopting low-level programming models like OpenCL or CUDA that must be
explicitly defined by the programmer in order to set up and release the device,
allocate space and move data to and from the device, and synchronize processing
at control flow barriers that logically span devices. This protocol logic
becomes more complex when tuning for performance by minimizing the number of
data transfers that occur and using asynchronous memory transfers to overlap
computation and data movement. In many standardized programming systems that are
used for programming hybrid computers, there is an intermingling of
computational and application logic with this platform-oriented protocol logic.
This leads to excessively complex and tedious software that is prohibitively
difficult to develop, maintain, and compose with other software.

\subsection{Protocol vs. Computational Logic}

For many problems, the \emph{computational logic} of a program related to the
goals of the application is fairly simple to state -- perform some set of
functions on vector or array data. Current accelerator programming techniques in
common languages for high performance computing (such as C, C++, and Fortran)
require this computational logic to be intermixed with the \emph{protocol logic}
to varying degrees, resulting in programs that are complex to write, maintain,
and tune for performance. Composition of independently developed program units
that use the hybrid architectures is similarly complex due to limited (or
nonexistent) methods for reasoning about the product of composition in terms of
lower level memory usage and task creation. In this paper, we will focus on the
specific instance of this problem presented by GPU-based accelerators.

We hypothesize that one can avoid the complexity that this intermingling of
computational and protocol logic without resorting to specialized programming
systems such as vendor-specific directive methods like the PGI
Accelerator~\cite{pgi-accelerate} or HMPP~\cite{hmpp} systems. An abstraction is
possible by extending the type system of a traditional language such that
protocol logic can be turned into a problem of reasoning about types alone.

%This pattern suggests that programmers should consider abstracting the
%interface to the GPU that constitutes the protocol logic of the
%program, so that they can focus on the domain oriented computational
%logic of the program. This will allow them to avoid obfuscating their
%computational logic with the tedious details of interacting with the
%device.  \comment{This paragraph is weak}

It is crucial that GPU programming systems are able to integrate with existing
code and programming tools. Scientists in the national laboratories, for
example, typically write code in popular general-purpose languages such as
C/C++, Fortran, or Python. They also make frequent use of the vast catalog of
libraries in these languages. Domain-specific accelerator languages which cannot
integrate, or integrate only with difficulty, with existing codes in these
languages are of limited use in these scenarios. Instead, a system that imposes
a minimal and orthogonal extension to the type system of an existing language
can aid users of existing languages without being forced to adopt a single
vendor-defined model.

\subsection{The Twig Approach}

In this paper we present a high-level programming language called \emph{Twig}
which has been designed for expressing protocol logic and for easy integration
with mainstream languages. We then show how Twig can be used to overcome some of
the obstacles to GPU accelerator programming described above. Twig's design
allows it to express GPU programming logic simply, and at the same time supports
automated reasoning about composite programs that can, in many cases, avoid
redundant memory copying and thereby retain high performance. Crucially, Twig's
role in the programming toolchain is to generate code in a mainstream language
like C. The generated code is then incorporated into a surrounding program, and
then built as usual. This minimizes the complexity that Twig adds to the build
process, and allows Twig code to interact easily with existing code and
libraries. Generated code induces no additional dependency on a third-party tool
since it can be integrated directly into the code base, requiring only the
compilation tools of the base language and standardized libraries (such as
OpenCL) for supporting hybrid processing.

Twig achieves these goals by leveraging the theory of term
rewriting~\cite{baader98rewriting}, along with a scheme which uses data types to
direct the generation of code in the target language. In particular, we augment
existing data types in the target language with a notion of \emph{location},
e.g., an array of floats located on a GPU, or an integer located in main memory.
In this paper, we do not present a full toolchain, but instead focus on the
crucial aspect of Twig upon which this work relies -- the translation to and
from a term representation of types and operations on these types, and the code
generation techniques used to map the result of term manipulation to the base
language and application code. We describe the basics of Twig's language in the
following section, and then discuss how we use this language along with located
types to generate code for GPUs. We also show how optimization can be performed
for data movement between devices via term rewriting formalisms.

% Writing code for accelerators such as GPUs can be tedious and error prone. One
% problem is that the dominant methods for programming in these environments,
% such as CUDA or OpenGL, provide only a low-level interface to the hardware.
% For many programmers, in particular domain programmers such as simulation
% scientists, a higher-level interface may be desired. The trade-off, however,
% may be a pronounced decrease in performance. Since performance is generally
% the salient reason to use an accelerator in the first place, this situation
% represents a major impediment to the widespread adoption of accelerator
% hardware environments by non-specialist programmers.

% One of the major potential impediments to performance on accelerators, of
% particular hazard to high-level programming models, is redundant memory
% movement. Accelerators typically possess their own special-purpose memory,
% separate from the main system memory. Programmers must explicitly copy data
% from the system to the accelerator before it can be processed there. Once the
% data is processed, it must be copied back into the system memory. Copying
% memory to and from the device is typically quite slow, presenting one of the
% major bottlenecks for accelerator performance. Programmers must be careful,
% therefore, to copy data only as needed, and to do as much processing as
% possible on the device before copying the data back to the system.
