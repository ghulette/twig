%!TEX root = twig-gpu.tex

\subsection{Code generation}
\label{twig:code-gen}

Twig provides an abstract model of code generation. We discuss our concrete
implementation, which generates C code, in
Section~\ref{twig:concrete-code-gen}. For our purposes, ``code'' is
represented by an abstract data type $M$. Each $x$ in $M$ is intended to
represent a ``block'' of code; i.e., something that performs some operation on
a set of inputs and produces a set of outputs. Although this is the intended
interpretation, our model elides the details and focuses only on the inputs
and outputs for each block, and how they can be arranged to form larger
blocks.

For all $x$ in $M$, we define two functions \texttt{in} and \texttt{out},
which map elements in $M$ to, respectively, the number of inputs and outputs
of the element. In either case, the number may be zero.

Given this structure, we provide two fundamental operations. The first is
\emph{sequential composition}, where the outputs of one code block are fed
into the inputs of another. We represent this operation as addition ($+$) and
say that

\[
\infer
{x+y \in M}
{x \in M \quad y \in M \quad \mathtt{out}(x) = \mathtt{in}(y)}
\]

and

\begin{eqnarray*}
\mathtt{in} (x+y) &=& \mathtt{in}(x)\\
\mathtt{out}(x+y) &=& \mathtt{out}(y)
\end{eqnarray*}

Note that the condition that the number of outputs on the left side equal the
number of inputs on the right implies that $M$ is not closed under sequential
composition. We take care in the design of the rest of Twig's semantics to
ensure that this is never a problem.

The second operation is \emph{parallel composition}, where two blocks execute
independently of one another. We represent this operation as multiplication
($\times$) and say that

\[
\infer
{x \times y \in M}
{x \in M \quad y \in M}
\]

and we define

\begin{eqnarray*}
\mathtt{in} (x \times y) =& \mathtt{in}(x)  + \mathtt{in}(y)\\
\mathtt{out}(x \times y) =& \mathtt{out}(x) + \mathtt{out}(y)
\end{eqnarray*}

Note that $M$ is closed under parallel composition.

We define a set of special elements in $M$, namely $\Pi_m(i_1,\ldots,i_n)$,
where $i_1,\ldots,i_n \in \lbrace i | 1 \leq i \leq m \rbrace$. These are
called \emph{permutation} elements, and represent the primitive operation of
rearranging $m$ inputs to $n$ outputs, possibly in a different order, and
possibly duplicating or dropping elements. The elements are only passed
through, and are otherwise unchanged.

Among the permutation elements, there are a set of elements for which we
provide special rules; namely, a set of \emph{identity} permutations. The
simplest of these is $\Pi_1(1)$, which acts as an identity transformation with
one input and one output. We refer to this element as $I_1$. In fact, there
are an unlimited number of identity transformations, which take $n$ inputs to
$n$ outputs, unchanged. These are referred to as $I_n$, where $1 \leq n$, and
$I_n = \Pi_n(1,2,\ldots,n)$. Clearly,

\[
\mathtt{in}(I_n) = \mathtt{out}(I_n) = n
\]

Since the elements $I_n$ are intended to represent identity operations, we
assign them a special meaning. Namely,

\[
\infer{x + I_n \to x}{\mathtt{out}(x) = n} 
\qquad
\infer{I_n + x \to x}{\mathtt{in}(x) = n}
\]

That is, $I_n$ acts as both a left- and right-identity when sequenced with a
block $x$. We sometimes use $I_n$ as a kind of ``no-op.''

It is worth noting one further identity, namely

\[
I_n = (I_1)^n
\]

That is, $I_n$ is equivalent to the $n$-way parallel composition of $I_1$.

When $n$ is implied from the context we will sometimes write $I$ for $I_n$.
For example, $x+I$ as a shorthand to refer to $x+I_n$ where $n =
\mathtt{out}(x)$.

Using this simple system for composition, a wide variety of code can be
generated.
