%!TEX root = twig-gpu.tex

\subsection{Code generation}
\label{twig:code-gen}

Twig provides an abstract model of code generation. We discuss our concrete
implementation, which generates C code, in
Section~\ref{twig:concrete-code-gen}. For our purposes, ``code'' is
represented by an abstract data type $M$. Each $x$ in $M$ is intended to
represent a ``block'' of code; i.e., something that performs some operation on
a set of inputs and produces a set of outputs. Although this is the intended
interpretation, our model elides the details and focuses only on the inputs
and outputs for each block, and how they can be arranged to form larger
blocks.

For all $x$ in $M$, we define two functions \texttt{in} and \texttt{out},
which map elements in $M$ to, respectively, the number of inputs and outputs
of the element. In either case, the number may be zero.

Given this structure, we provide two fundamental operations. The first is
\emph{sequential composition}, where the outputs of one code block are fed
into the inputs of another. We represent this operation as addition ($+$) and
say that

\[
\infer
{x+y \in M}
{x \in M \quad y \in M \quad \mathtt{out}(x) = \mathtt{in}(y)}
\]

and

\begin{eqnarray*}
\mathtt{in} (x+y) =& \mathtt{in}(x)\\
\mathtt{out}(x+y) =& \mathtt{out}(y)
\end{eqnarray*}

Note that the condition that the number of outputs on the left side equal the
number of inputs on the right implies that $M$ is not closed under sequential
composition. We take care in the design of the rest of Twig's semantics to
ensure that this is never a problem.

The second operation is \emph{parallel composition}, where two blocks execute
independently of one another. We represent this operation as multiplication
($\times$) and say that

\[
\infer
{x \times y \in M}
{x \in M \quad y \in M}
\]

and we define

\begin{eqnarray*}
\mathtt{in} (x \times y) =& \mathtt{in}(x)  + \mathtt{in}(y)\\
\mathtt{out}(x \times y) =& \mathtt{out}(x) + \mathtt{out}(y)
\end{eqnarray*}

Note that $M$ is closed under parallel composition.

We introduce a set of special elements in $M$, named $I_n$ where $n \geq 0$
and define

\begin{eqnarray*}
\mathtt{in} (I_n) =& 1\\
\mathtt{out}(I_n) =& n
\end{eqnarray*}

The block $I_n$ is intended to represent an identity operation, where the
single input is copied to the output or outputs. To support this
interpretation, we assign a special meaning to $I_1$. Namely,

\[
\infer{x + I_1 \to x}{x + I_1 \in M} 
\qquad
\infer{I_1 + x \to x}{I_1 + x \in M}
\]

That is, $I_1$ acts as a left and right identity when sequenced with a block
$x$, corresponding to the notion of a ``no-op.'' The requirement that the
sequence be in $M$ restricts $x$ in the rules such that it must have exactly
one output or one input, respectively.

The element $I_0$ is also special. It takes one input and produces zero
outputs, and represents a ``null pipe'' which discards its argument.

Using this simple system for composition, a wide variety of code can be
generated from a few primitives.
