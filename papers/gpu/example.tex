%!TEX root = twig-gpu.tex

\section{Example}

The following example shows how reductions in Twig can eliminate redundant
memory copies.

\begin{verbatim}
copyToGPU   = [(array(X),int)     -> (gpuarray(X),int)] {...}
copyFromGPU = [(gpuarray(X),int)  -> (array(X),int] {...}
kernel{k}   = [(gpuarray(X),int)) -> (gpuarray(X),int)] {
  ...
  invoke_kernel($k);
  ...
}
execKernel(k) = copyToGPU;kernel(k);copyFromGPU
main          = execKernel{"foo"};execKernel{"bar"}

@reduce (copyFromGPU;copyToGPU) => id	
\end{verbatim}

This example is considerably simplified in the interest of brevity and clarity.
We mark the missing C and OpenCL code for each rule with an ellipsis.

The first three lines define primitive rules for moving data to and from the
GPU, and for invoking a kernel transformation on the array in GPU memory. The
kernel rule is parameterized with an argument \texttt{k}, which allows us to
pass the name of the kernel to be executed.

The \texttt{execKernel} rule will perform one kernel operation on the GPU. Note
that this rule will be semantically valid in whatever context it is called,
since it first moves the data onto the GPU, then executes the kernel, and then
copies it back (in a more realistic program, we might block while the copies
complete). The semantics are those of a function on an array -- considerably
simpler than the abstraction presented by OpenCL or CUDA alone.

The \texttt{main} rule is the top level of the program. It executes two kernels
in sequence with two invocations of \texttt{execKernel}. As noted above, each
invocation would normally result in a copy to and from the GPU. Since the data
is not modified in between GPU calls, on its own this function would generate a
redundant copy.

The next line, however, specifies a reduction which will eliminate the redundant
copy. It simply says that if data is copied from the GPU to the system and then
immediately back, we can replace the copying with an identity transformation,
i.e., a no-op.

Although this example is quite simple, it demonstrates the power of reductions.
The reduction rule given here would probably be paired with the \texttt{copyTo}
and \texttt{copyFrom} rules in a module that could be exposed for domain
programmers, allowing them to perform GPU operations without ever worrying about
the rule design. Sophisticated users, however, could even add
application-specific reductions, enabling very powerful and customizable code
generation based on domain-specific logic.
