%!TEX root = twig-gpu.tex

\section{Located Types}

An important principle of Twig's design is the ability to augment type
information. Twig generates C code, and that code must therefore conforms to C's
type system. However, Twig can operate on more complex types, as long as these
types have a mapping to C. This enables us to design primitive rules that insist
on stricter type information than would be available in C alone.

For GPU programming, we exploit this capability by adding a notion of
\emph{location} to the usual C types. Location in this case describes where the
data is stored in memory, i.e., either in system memory or on the GPU. For
example, an array of integers in system memory is represented by the simple term
\texttt{array(int)}, while the same data on the GPU is represented by the term
\texttt{gpu(array(int))}. By wrapping the basic data type with the location
information, we ensure that rules must be specific to the GPU in order to
operate on GPU data. For example,

\begin{verbatim}
[gpu(array(float)) -> gpu(array(int))] {...}
\end{verbatim}

will convert an array of floats to an array of integers, but only if the data
already resides on the GPU. If it does not, the data must be moved with a rule
such as

\begin{verbatim}
[array(float) -> gpu(array(float))] {...}
\end{verbatim}

This scheme could be extended to support more complex locations, such as
multiple GPUs.
