\documentclass{acm_proc_article-sp}
% \usepackage{fullpage}
\usepackage{amssymb}
\usepackage{url}
\usepackage{parskip}

\title{Twig for GPUs}
\author{Geoffrey C. Hulette, Matthew Sottile, Allen D. Maloney}  

\begin{document}
\maketitle
\thispagestyle{empty}

\section{Introduction}

Programmers are often constrained by engineering requirements to produce work
in a mainstream language such as C or Java. At the same time, there is an
increasing awareness of the benefits of \emph{domain-specific} programming
languages (DSLs) for letting programmers focus on the problem they are solving
instead of tedious fine-grained implementation details. The goal of DSLs is to
closely model the problem domain that a programmer is working within, and this
brings a number of benefits. First, the reduction in abstraction from
algorithm statement to a corresponding implementation that is required by
traditional languages is avoided, allowing them to be expressed clearly and
succinctly in a DSL. Second, DSLs may admit automated reasoning about
programs, such as optimization for efficiency in running time or space
requirements, or proving that the program adheres to required constraints such
as variable value ranges. Where a DSL provides these benefits and also
generates C (or some other mainstream language) as output, it may be useful
and relatively easy to integrate with existing engineering processes.
Furthermore, domain specific languages reside at a higher level of abstraction
than traditional languages, allowing them to be more easily retargeted to new
architectures due to a lack of specificity regarding how high level algorithms
should be mapped to specific architectures. This final benefit is particularly
attractive for non-traditional processor architectures emerging in the market
at present.

The power of DSLs comes at a cost, however. DSLs typically have a unique
syntax and semantics that may be known primarily to the language's author. The
interpreter and other tools may be difficult to support or extend if the
original author becomes unavailable. It can also be difficult to extend DSLs
to support new constructs or rules. This has been a long standing argument in
favor of continued preference for traditional general purpose programming
languages.

%% Matt - this paragraph needs more work.  Why does twig help fix the problems
%% pointed out above?

We introduce Twig, a DSL intended for ``ad-hoc'' modeling of a variety of
high-level program semantics, and admitting simple reasoning via equivalence
rules. The evaluation of a Twig program generates code in C (and potentially
other languages), lending it to integration in software existing projects. By
``ad-hoc'' we mean that Twig can be configured rather easily by application
programmers to serve as a DSL for different purposes, even within the same
application. 


This may alleviate some of the resistance to the adoption of Twig-based DSLs
in projects.

In this paper, we demonstrate how Twig can be used to build a domain specific
extension to traditional languages like C to address what we believe are
limitations to existing type systems that complicate the programming of hybrid
architectures such as those based on GPGPUs. All mainstream languages are
based on an assumption that the underlying architecture, either via hardware
assistance or software runtime support, provides the illusion of a single
addressable memory accessible by all cores within a computer. Hybrid systems
violate this by partitioning memory between processors of different types
(such as a CPU and GPU), and forcing data movement between these memories to
be explicitly stated. Traditional languages must be extended to abstract away
the management of this data movement. This paper shows how Twig can be used to
move this into the type system of a traditional language via a domain specific
extension. % Too wordy, but ...

\section{Example Problem: GPGPU Programming}

Programming for GPUs can be a challenging task. There is a considerable amount
of protocol that must be invoked in order to set up and release the device,
move data to and from the device, synchronize processing at barriers. Yet for
many problems, the \emph{computational logic} of a program is fairly simple to
state -- perform some function on a vector or array. Current GPGPU programming
techniques require this computational logic to be intermixed with the
\emph{protocol logic}, resulting in programs that are complex to write,
maintain, and tune for performance. Composition of independently developed
program units that use the GPGPU is similarly complex due to limited (or
nonexistent) methods for reasoning about the result of this composition in
terms of lower level memory usage and task creation.

This pattern suggests that programmers should consider abstracting the
interface to the GPU that constitutes the protocol logic of the program, so
that they can focus on the domain oriented computational logic of the program.
This will allow them to avoid obfuscating their computational logic with the
tedious details of interacting with the device. There are numerous ways to
approach this problem.

First, programmers could build a library of functions or objects which hide as
many details as possible. In fact this has already been done to a large
extent, with frameworks such as CUDA or OpenCL, and these libraries are the
primary way in which programmers already program for GPUs. This approach
mitigates many difficult issues, but provides only a minimal abstraction --
the library abstraction cannot hide the need to coordinate GPU activities such
as set up and memory management. Programmers might try to create their own
libraries on top of CUDA or OpenCL, but in most languages the object or
library facilities will not allow them to reason at a high semantic level. In
other words, they might be able to simplify some operations by specializing
them (e.g. ignore the possibility for multiple devices), but they cannot hide
them altogether.

Another approach is to design a domain-specific language for GPUs. Examples
include PyCUDA and OpenMPC. Domain-specific languages are typically not
customizable for higher-level application logic.

\subsection{The Twig Approach}

% What is the twig approach?  Give a little info here.

A key milestone in the history of programming languages was the introduction
of high level types into early languages. The ability of programmers to
separate the representation of a value of different types (e.g., floating
point versus integer; record types; arrays) allowed program code to be written
in terms of the mathematical abstractions that bits in memory represented
without exposing how the values were laid out in memory. We believe that a
similar type-level abstraction is possible in modern hybrid computer
architectures. The abstraction that must be moved to the type system is that
of the location of data within the system.

Changes in representation of basic values, such as integers or floats, is
often as simple as a type casting operator or an implicit type conversion as
allowed by the language. If residence information about data is part of the
type system, then movement of data within the memory hierarchy of a hybrid
machine can also be made as simple to express as traditional type coercions.
For the domain of GPGPU programming, Twig allows a domain specific type system
to be created in which this additional location information can be added to
the type of variables with corresponding type coercion operators that map to
lower level memory movement logic that defines the protocol between the host
and GPU device.

In addition, by abstracting the protocol operations away to high level type
coercion operators, Twig can support automated reasoning about programs that
allows optimizations to be performed related to the underlying protocol logic.
For example, if a programmer defines a sequence of statements that, based on
their type, state that data will move back and forth between the host and
device without modification on the host side, unnecessary copies of data can
be removed. This will increase program performance by reducing unnecessary
burden on the memory subsystem.

Current systems like CUDA and OpenCL allow the programmer to explicitly
allocate memory on one of the many devices that control some portion of the
memory within a machine, but require copies to be explicitly implemented by
the programmer. The result is fine grained control of the machine at the cost
of limited analysis and optimization of data movement within the memory
hierarchy. This makes writing complex programs difficult, particularly with
respect to performance. We believe that a Twig-based DSL for GPU programming
based on extending the type system of existing languages is a step towards
removing this burden from the programmer.

\section{Related Work}

In all cases, why are they insufficient or why don't they address the problem
that twig addresses?

Discuss PyCUDA and OpenMPC. What about Jacket and other GPU-type interfaces
for languages like Matlab?

HPF was interesting because it explored extending the type system of Fortran
90 to add data distribution information, which feels similar to the Twig GPU
idea. definitely need to talk about that at some point in the paper.

Other general-purpose DSLs? Yampa/Arrows?

Accelerate GPU language in Haskell is similar.  

\section{Design of the GPU DSL}

Details on the GPU DSL.  This should be used to set up the next section so
that the reason twig is talked about is clear.

\section{Twig}

Overview of how Twig works, basic semantics, code generation, simple example.

Focus on importance of generating C - integration with existing methods, type-directed generation.

\subsection{Reductions}

Explain reductions and how they can be used in domain- or application-specific
ways.

Explain built-in reductions. Maybe discuss idea of a normal form, although
this might be better in a later paper.

\section{Implementation}

Talk briefly about how Twig is implemented, and how to use it to generate
code.

\section{Evaluation}

Show some GPU algorithm encoded in Twig. Discuss advantages over simple APIs
(restricted flexibility equals improved ability to reason).

Show how reductions eliminate redundant copies.

Compare performance versus naive generation.

Also compare against PyCUDA or another system?

\section{Future work}

Future work.

\section{Conclusion}


% \section{Section Title}
% \input{section_file}

% \pagebreak
% \bibliographystyle{plain}
% \bibliography{references}

\end{document}
