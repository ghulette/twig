\documentclass{acm_proc_article-sp}
% \usepackage{fullpage}
\usepackage{amssymb}
\usepackage{url}
\usepackage{parskip}

\begin{document}

\title{Twig for GPUs}

\numberofauthors{3}

\author{
% 1st. author
\alignauthor
Geoffrey C. Hulette\\
\affaddr{University of Oregon}\\
\affaddr{Eugene, OR}\\
\email{ghulette@cs.uoregon.edu}
% 2nd. author
\alignauthor
Matthew Sottile\\
\affaddr{Galois, Inc.}\\
\affaddr{Portland, OR}\\
\email{mjsottile@computer.org}
% 3rd author
\alignauthor
Allen D. Malony\\
\affaddr{University of Oregon}\\
\affaddr{Eugene, OR}\\
\email{malony@cs.uoregon.edu}
}  

\maketitle

\begin{abstract}

Numerous programming models have been introduced in the last few years to
allow programmers to utilize new accelerator-based architectures. While OpenCL
and CUDA represent similar low-level abstractions for programming
accelerators, most often based on GPU processors, few higher level programming
models have emerged that are compatible with common general purpose languages.
In this paper, we propose that extensions to the type systems of traditional
general purpose languages can be made that allow programmers to continue to
work at a high level of abstraction with respect to memory, deferring much of
the tedium of data management and movement code in OpenCL or CUDA to an
automatic compilation tool. Furthermore, the techniques based on formal term
rewriting that are described in this paper take advantage of reduction and
simplification rules that can be used to optimize memory optimizations to
reduce unnecessary storage requirements and transfer activity.

\end{abstract}

%!TEX root = twig-gpu.tex

\section{Introduction}

Writing code for accelerators such as GPUs can be tedious and error prone. One
problem is that the dominant methods for programming in these environments,
such as CUDA or OpenGL, provide only a low-level interface to the hardware.
For many programmers, in particular domain programmers such as simulation
scientists, a higher-level interface may be desired. The trade-off, however,
may be a pronounced decrease in performance. Since performance is generally
the salient reason to use an accelerator in the first place, this situation
represents a major impediment to the widespread adoption of accelerator
hardware environments by non-specialist programmers.

One of the major potential impediments to performance on accelerators, of
particular hazard to high-level programming models, is redundant memory
movement. Accelerators typically possess their own special-purpose memory,
separate from the main system memory. Programmers must explicitly copy data
from the system to the accelerator before it can be processed there. Once the
data is processed, it must be copied back into the system memory. Copying
memory to and from the device is typically quite slow, presenting one of the
major bottlenecks for accelerator performance. Programmers must be careful,
therefore, to copy data only as needed, and to do as much processing as
possible on the device before copying the data back to the system.

Another problem with high-level accelerator programming environments is that
of integration with existing and supporting code. Scientists at Sandia, for
example, typically write code in popular general-purpose languages such as
C/C++, Fortran, or Python. They also make frequent use of the vast catalog of
libraries in these same languages. Domain-specific accelerator languages which
cannot integrate, or integrate only with difficulty, with existing codes in
these languages are of limited use in these scenarios.

In this paper we present a high level code generation language called
\emph{Twig} and show how it can be used to overcome the obstacles to
high-level accelerator programming described above. In particular, Twig allows
for simple, high-level accelerator programming, and at the same time supports
automated reasoning about composite programs that can, in many cases, avoid
redundant memory copying and thereby retain high performance. Finally, Twig's
role in the programming toolchain is to generate code in a mainstream language
like C. The generated code is then incorporated into a surrounding program,
which is built as usual. This minimizes the complexity that Twig adds to the
build process, and allows Twig code to interact easily with existing code and
libraries.

Twig achieves these goals through a formal basis in term rewriting, along with
a scheme which uses data types to direct the generation of code in the target
language. In particular, we augment existing data types in the target language
with a notion of \emph{location}, e.g., an array of floats located on a GPU,
or an integer located in main memory. We describe the details of this approach
in the following section.

% Why C? Because it is lingua franca, and it is the language of accelerator %
% APIs.

%!TEX root = twig-gpu.tex

\section{Twig}

% Overview of how Twig works, basic semantics, code generation, simple
% example. Focus on importance of generating C - integration with existing
% methods, type-directed generation.

Twig is based on a core semantics called System S~\cite{Visser:1998p333}.
System S was originally designed for specifying term rewriting systems
(described briefly in \ref{section:term-rewriting}). In Twig, we use the
operators of System S to combine primitive \emph{rules} into more complex
transformations on types. These transformations are then applied to a given
type, which performs the transformation. We extend the semantics of System S
in order to have the evaluation of the transformation generate code as a side
effect. In this way, domain specific code can be generated depending on the
input types. In this section we describe the basics of Twig's language.

\subsection{Code generation}

Twig provides a basic but effective set of operations to perform code
generation. Our model is abstract, although we also discuss our concrete
implementation below. For our purposes, ``code'' is an abstract data type,
with two attributes: a list of inputs and a list of outputs. Each input and
output is described by a pair of a unique name, and also a term (described
below). There are two operations that are defined on code.

First

\subsection{Values}

Values in Twig can be any valid \emph{term}. Terms are tree structured data
with labeled internal nodes. Examples of terms include simple values like
\texttt{int} and \texttt{float}, as well as compound types like
\texttt{ptr(int)}, which represents a pointer to an integer.

Note that terms in Twig are intended to represent types in the target
language. The programmer can control the mapping via a configuration file.
Furthermore, the mapping need not be injective, i.e. you can have multiple
values in Twig map to the same type in C. For example, you might use distinct
Twig values \texttt{string} and \texttt{ptr(char)}, but have both represent a
\texttt{char *} in C.

In Twig, terms may also represent tuples. Tuples are represented with either a
special constructor \texttt{tuple}, or equivalently (and preferably) with no
constructor at all. For example, \texttt{(string,int)} (or, equivalently
\texttt{tuple(string,int)}) represents a pair of two types in C, a
\texttt{string} and an \texttt{int}. The \texttt{tuple} constructor syntax is
convenient mainly for our presentation of formal semantics for tuple
operators, given below.

\subsection{Rules}

The basic building blocks of a Twig program are called \emph{rules}. A rule
describes a basic transformation between terms. For example, in C it is easy
to convert an integer to a float. In Twig, this rule is written

\begin{verbatim}
[int -> float]
\end{verbatim}

The term to the left of the arrow is the input, and the term to the right is
the output. In this example, the rule says that if the input value is
\texttt{int} then it will be rewritten to the value \texttt{float}. If the
input is not \texttt{int} then it will be rewritten to the special value
$\bot$, which can be read as ``failure.''

Rules can also have \emph{variables} in place of terms or sub-terms. For
example the rule

\begin{verbatim}
[ptr(X) -> X]
\end{verbatim}

rewrites any pointer type to its referent.

\subsection{Combinators}

Rules can be combined into more complex expressions using a set of provided
operators. In the formal semantics, let $t$ range over terms, and $s_i$ range
over rule expressions (i.e. a primitive rule, or another rule expression built
with operators). Recall that $\bot$ denotes failure of rule application.

The most important operator is \emph{sequence}, which chains the application
of two rules together, feeding the output of the first into the input of the
second, and failing if either rule fails (see
Figure~\ref{semantics:sequence}). With this operator, simple rules can be
composed into multi-step transformations.

\begin{figure}[ht]
\label{semantics:sequence}
\[
\infer{t \arr{s_1;s_2} t''}{t \arr{s_1} t' \quad t' \arr{s_2} t''}
\qquad 
\infer{t \arr{s_1;s_2} \bot}{t \arr{s_1} \bot}
\qquad
\infer{t \arr{s_1;s_2} \bot}{t \arr{s_1} t' \quad t' \arr{s_2} \bot}
\]
\caption{Semantics for sequence operator}
\end{figure}

Another important binary operator is \emph{left-biased choice}. This operator
will try the first rule expression, and if it succeeds then its output is the
result (see Figure~\ref{semantics:choice}). If it fails (i.e. results in
$\bot$), then the second rule is tried. This operator allows different paths
to be taken, and different code to be generated, depending on the input type
being passed.

\begin{figure}[ht]
\label{semantics:choice}
\[
\infer{t \arr{s_1|s_2} t'}{t \arr{s_1} t'}
\qquad 
\infer{t \arr{s_1|s_2} t'}{t \arr{s_1} \bot \quad t \arr{s_2} t'}
\qquad
\infer{t \arr{s_1|s_2} \bot}{t \arr{s_1} \bot \quad t \arr{s_2} \bot}
\]
\caption{Semantics for left-biased choice}
\end{figure}

Figure~\ref{semantics:basic} gives the formal semantics for Twig's other
operators. These include constant operators, operators which discard their
results, and a fix point operator for recursion.

\begin{figure}[ht]
\label{semantics:basic}
\[
\infer{t \arr{\mathtt{id}} t}{}
\qquad
\infer{t \arr{\mathtt{fail}} \bot}{}
\]

\[
\infer{t \arr{?s} t}{t \arr{s} t'}
\qquad 
\infer{t \arr{?s} \bot}{t \arr{s} \bot}
\qquad
\infer{t \arr{\lnot s} \bot}{t \arr{s} t'}
\qquad 
\infer{t \arr{\lnot s} t}{t \arr{s} \bot}
\]

\[
\infer{t \arr{\mu x(s)} t'}{t \arr{s[x := \mu x(s)]} t'}
\qquad 
\infer{t \arr{\mu x(s)} \bot}{t \arr{s[x := \mu x(s)]} \bot}
\]
\caption{Semantics for basic operators}
\end{figure}

Twig also provides some special operators for tuples. The semantics in
Figure~\ref{semantics:all-tuples} apply to all the tuple operators, and state
simply that tuple operators will fail if the input term is not a tuple, or if
the rule references a tuple element out of bounds.

\begin{figure}[ht]
\label{semantics:all-tuples}
\[
\infer{f(\ldots) \arr{s} \bot}
{f \neq \mathtt{tuple}}
\qquad
\infer{\mathtt{tuple}(t_1,\ldots,t_n) \arr{s(i)} \bot}{i > n}
\]
\caption{Common semantics for tuple operators}
\end{figure}

One important tuple operator is \emph{congruence}, which applies a tuple of
rules to a tuple of values, and returns a tuple of results or failure if any
one rule application fails. The semantics are given in
Figure~\ref{semantics:congruence}.

\begin{figure}[ht]
\label{semantics:congruence}
\[
\infer{
\mathtt{tuple}(t_1,\ldots,t_n)
\arr{(s_1,\ldots,s_n)}
\mathtt{tuple}(t_1',\ldots,t_n') }
{t_1 \arr{s_1} t_1' \quad \ldots \quad t_n \arr{s_n} t_n'}
\]

\[
\infer{
\mathtt{tuple}(\ldots,t_i,\ldots)
\arr{(\ldots,s_i,\ldots)}
\bot}
{t_i \arr{s_i} \bot}
\]
\caption{Semantics for congruence operator}
\end{figure}

Projection:

\[
\infer{\mathtt{tuple}(\ldots,t_i,\ldots) \arr{\#i} t_i}{}
\]


Path:

\[
\infer{\mathtt{tuple}(\ldots,t_i,\ldots) \arr{\#i(s)} 
\mathtt{tuple}(\ldots,t_i',\ldots)}
{t_i \arr{s} t_i'}
\]

\[
\infer{\mathtt{tuple}(\ldots,t_i,\ldots) \arr{\#i(s)} \bot}
{t_i \arr{s} \bot}
\]

Branch:

\[
\infer{
  t \arr{\mathtt{\#branch}(s_1,\ldots,s_n)}
  \mathtt{tuple}(t_1',\ldots,t_n')
}{t \arr{s_1} t_1' \quad \ldots \quad t \arr{s_n} t_n'}
\]

\[
\infer
{t \arr{\mathtt{\#branch}(\ldots,s_i,\dots)} \bot}
{t \arr{s_i} \bot}
\]

Map:

\[
\infer{
  \mathtt{tuple}(t_1,\ldots,t_n)
  \arr{\mathtt{\#map}(s)}
  \mathtt{tuple}(t_1',\ldots,t_n')
}{t_1 \arr{s} t_1' \quad \ldots \quad t_n \arr{s} t_n'}
\]

\[
\infer
{\mathtt{tuple}(\ldots,t_i,\ldots) \arr{\mathtt{\#map}(s)} \bot}
{t_i \arr{s} \bot}
\]

\subsection{Code generation}

Our implementation of Twig generates C code, but in principle almost any
language could be generated instead.

\subsection{Reductions}

Reductions are a way to transform Twig expressions. Reductions may exploit
some application or domain knowledge about the nature of the rules, and as
such are usually developed alongside a set of rules.

% Explain reductions and how they can be used in domain- or
% application-specific ways.

\subsection{Term rewriting}
\label{section:term-rewriting}

% Brief overview of term rewriting and how it is used in reductions.

\subsection{Implementation}

% Talk briefly about how Twig is implemented, and how to use it to generate
% code.

%!TEX root = twig-gpu.tex

\section{Accelerator Programming}

Programming for GPUs can be a challenging task, in large part due to the
partitioned memory model that they impose on programmers. Unlike a basic SMP,
data must be explicitly moved within the memory hierarchy such that the
appropriate processing device can access it. There is a considerable amount of
protocol that must be invoked in order to set up and release the device, move
data to and from the device, and synchronize processing at barriers. This
protocol logic becomes more complex when tuning for performance by using
asynchronous memory transfers to overlap computation and data movement. Yet
for many problems, the \emph{computational logic} of a program related to the
goals of the application is fairly simple to state -- perform some set of
functions on vector or array data. Current GPU programming techniques in
common languages for high performance computing (such as C, C++, and Fortran)
require this computational logic to be intermixed with the \emph{protocol
logic}, resulting in programs that are complex to write, maintain, and tune
for performance. Composition of independently developed program units that use
the GPU is similarly complex due to limited (or nonexistent) methods for
reasoning about the result of this composition in terms of lower level memory
usage and task creation.

This pattern suggests that programmers should consider abstracting the
interface to the GPU that constitutes the protocol logic of the program, so
that they can focus on the domain oriented computational logic of the program.
This will allow them to avoid obfuscating their computational logic with the
tedious details of interacting with the device. There are numerous ways to
approach this problem.

First, programmers could build a library of functions or objects which hide as
many details as possible. In fact this has already been done to a large
extent, with frameworks such as CUDA or OpenCL, and these libraries are the
primary way in which programmers already program for GPUs. This approach
mitigates many difficult issues, but provides only a minimal abstraction --
the library abstraction cannot hide the need to coordinate GPU activities such
as set up and memory management. Programmers might try to create their own
libraries on top of CUDA or OpenCL, but in most languages the object or
library facilities will not allow them to reason at a high semantic level. In
other words, they might be able to simplify some operations by specializing
them (e.g. ignore the possibility for multiple devices), but they cannot hide
them altogether.

Another approach is to design a domain-specific language for GPUs. Examples
include PyCUDA and OpenMPC. Domain-specific languages are typically not
customizable for higher-level application logic.

A technique that is becoming popular and is based on the OpenMP model of
directive-based code annotations hides much of the protocol logic by leaving
it to a compiler and runtime system to implement it. The PGI
Accelerate~\cite{pgi-accelerate} model and the HMPP programming
system~\cite{hmpp} are examples of this in compiler products. This approach
will, like OpenMP, make significant progress on lowering the barrier of entry
for programmers. Unfortunately, these methods provide too much a black-box to
the programmer, limiting their ability to customize the underlying GPU
protocol logic.

\subsection{The Twig Approach}

% What is the twig approach?  Give a little info here.

A key milestone in the history of programming languages was the introduction
of high level types into early languages. The ability of programmers to
separate the representation of a value of different types (e.g., floating
point versus integer; record types; arrays) allowed program code to be written
in terms of the mathematical abstractions that bits in memory represented
without exposing how the values were laid out in memory. We believe that a
similar type-level abstraction is possible in modern hybrid computer
architectures. The abstraction that must be moved to the type system is that
of the location of data within the system.

Changes in representation of basic values, such as integers or floats, is
often as simple as a type casting operator or an implicit type conversion as
allowed by the language. If residence information about data is part of the
type system, then movement of data within the memory hierarchy of a hybrid
machine can also be made as simple to express as traditional type coercions.
For the domain of GPGPU programming, Twig allows a domain specific type system
to be created in which this additional location information can be added to
the type of variables with corresponding type coercion operators that map to
lower level memory movement logic that defines the protocol between the host
and GPU device.

In addition, by abstracting the protocol operations away to high level type
coercion operators, Twig can support automated reasoning about programs that
allows optimizations to be performed related to the underlying protocol logic.
For example, if a programmer defines a sequence of statements that, based on
their type, state that data will move back and forth between the host and
device without modification on the host side, unnecessary copies of data can
be removed. This will increase program performance by reducing unnecessary
burden on the memory subsystem.

Current systems like CUDA and OpenCL allow the programmer to explicitly
allocate memory on one of the many devices that control some portion of the
memory within a machine, but require copies to be explicitly implemented by
the programmer. The result is fine grained control of the machine at the cost
of limited analysis and optimization of data movement within the memory
hierarchy. This makes writing complex programs difficult, particularly with
respect to performance. We believe that a Twig-based DSL for GPU programming
based on extending the type system of existing languages is a step towards
removing this burden from the programmer.

%!TEX root = twig-gpu.tex

\section{Related Work}

Numerous systems have been created in recent years that provide an abstraction above low-level interfaces such as OpenCL or CUDA. These include the PGI Accelerate model\cite{pgi-accelerate} and the HMPP programming system\cite{hmpp}. While both of these systems provide an effective high-level abstraction, they offer little room for tuning the low-level interface to the accelerator. Twig provides a simple method for user-definable rewriting of programs, which allows architecture-, domain- and even application-specific optimizations to be realized.

Furthermore, in large applications it is infeasible to assume that all developers of the various components will use the same high-level abstraction. This makes program composition challenging, since it may be unclear how the objects generated by independent programming systems interacts. Twig adopts a code generation approach in which a single, low-level target (such as CUDA) is used. This approach solves the composability problem, since all Twig code maps to a single ``lingua franca'' for programming the hybrid system.

Twig was inspired in part by FIG\cite{fig}. In that project, a similar formal approach was used to express bindings between different programming languages. In our experience, multi-language programming has much in common with programming hybrid systems. The overlaps include memory ownership and management, data marshalling, and managing the flow of program control across the language or device boundary. Our work builds upon the approach in Fig, and in particular aims to provide a general-purpose tool not tied to the Moby\cite{moby} programming language.

Kennedy's \emph{telescoping languages} work is related to ours in that it seeks to support high-level programming by deferring challenging problems to automated tools. In particular, telescoping languages provide the ability to build domain-specific languages using scripting languages to coordinate functionality in domain-specific libraries\cite{kennedy01telescoping}. The strategy used to optimize these programs is quite similar in some ways to our approach with reductions (see Sec.~\ref{sec:reductions}).

Code generation approaches have had notable success in the computational science field, an exemplar being the Tensor Contraction Engine (TCE)\cite{baumgartner05synthesis}. The TCE allows computational chemists to write tensor contraction operations in a high level language, and then generates the corresponding collections of loops that implement the operations. Unlike Twig, the TCE is quite specialized, being of use only to programmers working with tensor-based computations.

% Also mention: Sequoia, Offload

\section{Design of the GPU DSL}

Details on the GPU DSL.  This should be used to set up the next section so
that the reason twig is talked about is clear.

%!TEX root = twig-gpu.tex

\section{Evaluation}

Show some GPU algorithm encoded in Twig. Discuss advantages over simple APIs
(restricted flexibility equals improved ability to reason).

Show how reductions eliminate redundant copies.

Compare performance versus naive generation.

Also compare against PyCUDA or another system?

\section{Conclusion}

\subsection{Future work}

Future work.


\section{Acknowledgements}


This work was supported in part by the Department of Energy Office of Science,
Advanced Scientific Computing Research.

% \pagebreak
\bibliographystyle{abbrv}
\bibliography{twig-gpu}

\end{document}
