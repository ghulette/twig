\documentclass{acm_proc_article-sp}
% \usepackage{fullpage}
\usepackage{amsmath}
\usepackage{amssymb}
\usepackage{proof}
\usepackage{url}
\usepackage{parskip}

\newcommand{\arr}[1]{\ensuremath\xrightarrow{#1}}

\begin{document}

\title{Twig for GPUs}

\numberofauthors{3}

\author{
% 1st. author
\alignauthor
Geoffrey C. Hulette\\
\affaddr{University of Oregon}\\
\affaddr{Eugene, OR}\\
\email{ghulette@cs.uoregon.edu}
% 2nd. author
\alignauthor
Matthew Sottile\\
\affaddr{Galois, Inc.}\\
\affaddr{Portland, OR}\\
\email{mjsottile@computer.org}
% 3rd author
\alignauthor
Allen D. Malony\\
\affaddr{University of Oregon}\\
\affaddr{Eugene, OR}\\
\email{malony@cs.uoregon.edu}
}

\maketitle

\begin{abstract}

Numerous programming models have been introduced in the last few years to
allow programmers to utilize new accelerator-based architectures. While OpenCL
and CUDA represent similar low-level abstractions for programming
accelerators, most often based on GPU processors, few higher level programming
models have emerged that are compatible with common general purpose languages.
In this paper, we propose that extensions to the type systems of traditional
general purpose languages can be made that allow programmers to continue to
work at a high level of abstraction with respect to memory, deferring much of
the tedium of data management and movement code in OpenCL or CUDA to an
automatic compilation tool. Furthermore, the techniques based on formal term
rewriting that are described in this paper take advantage of reduction and
simplification rules that can be used to optimize memory optimizations to
reduce unnecessary storage requirements and transfer activity.

\end{abstract}

%!TEX root = twig-language.tex

\section{Introduction}

Twig is a new language for writing \emph{typemaps} -- programs that transform data from one type to another, while preserving (as much as possible) the underlying value of the data. Typemaps have proven useful in many kinds of programming and especially automated code generation, where we require a transformation to pass a single nominal value across a pair of mismatched types that we know to be interchangeable in some way. The best-known example of this problem is found in multi-language programming. For example, a programmer may wish to pass a Python integer to a C function, where a C int is expected. If we have a typemap that performs the transformation from Python integers to C integers, then an automated tool can generate the conversion code automatically, and expose the C function in Python via a generated wrapper.

There are a number of existing tools and languages for creating typemaps and generating code from them. Twig builds on existing typemap tools in several ways.

First, Twig's typemaps are composable, i.e., new typemaps may be constructed by combining old ones. Thus, complex typemap transformations may be built from simpler ones. Our notion of typemap composition is based on the formalisms used in Fig\cite{fig} and System S\cite{system-s}, but we extend and refine that work in some key ways.

Second, Twig incorporates a robust, formal model of code generation. This allows Twig to generate code based on typemaps in many different target languages.

Finally, Twig includes a facility for \emph{reducing} typemaps by exploiting identity relationships among typemap expressions. Some reductions are based on a universal ``algebra of typemaps,'' while others are domain-specific and must be described by the user. We have shown in prior work that typemap reduction can be used to optimize certain transformations. Reductions are covered in our previous work, and we will not address them further here.

In this paper, we will describe Twig's formal language structure, and then show how this structure allows us to express complex typemaps more concisely than with traditional tools. First, we review existing approaches to typemaps and related problems. Second, we present the semantics for Twig's code generation model, and then the semantics for the typemap language itself. Third, we present a typemap example in Swig, and show how the same problem can be solved more concisely and clearly in Twig. Finally, we conclude with ideas for future work.

%!TEX root = twig-gpu.tex

\section{Twig}

% Overview of how Twig works, basic semantics, code generation, simple
% example. Focus on importance of generating C - integration with existing
% methods, type-directed generation.

Twig is based a core semantics called System S~\cite{Visser:1998p333}. System
S was originally designed for specifying term rewriting systems (described
briefly in \ref{section:term-rewriting}). In Twig, we use the operators of
System S to combine primitive \emph{rules} into more complex transformations
on types. These transformations are then applied to a given type in order to
generate code. In this section we describe Twig's language.

\subsection{Rules}

The basic building blocks of a Twig program are called \emph{rules}. A rule simply specifies that a given type can be transformed into another type, and provides a snippet of code which performs the transformation. The snippets are essentially strings

\subsection{Values}

Values in Twig are types in the target language.

\subsection{Combinators}

Rules can be combined into more complex expressions using a fixed set of
operators.

\subsection{Code generation}

Our implementation of Twig generates C code, but in principle almost any
language could be generated instead.

\subsection{Reductions}

Reductions are a way to transform Twig expressions. Reductions may exploit
some application or domain knowledge about the nature of the rules, and as
such are usually developed alongside a set of rules.

% Explain reductions and how they can be used in domain- or
% application-specific ways.

\subsection{Term rewriting}
\label{section:term-rewriting}

% Brief overview of term rewriting and how it is used in reductions.

\subsection{Implementation}

% Talk briefly about how Twig is implemented, and how to use it to generate
% code.

%!TEX root = twig-gpu.tex

\section{Accelerator Programming}

Programming for GPUs can be a challenging task, in large part due to the
partitioned memory model that they impose on programmers. Unlike a basic SMP,
data must be explicitly moved within the memory hierarchy such that the
appropriate processing device can access it. There is a considerable amount of
protocol that must be invoked in order to set up and release the device, move
data to and from the device, and synchronize processing at barriers. This
protocol logic becomes more complex when tuning for performance by using
asynchronous memory transfers to overlap computation and data movement. Yet
for many problems, the \emph{computational logic} of a program related to the
goals of the application is fairly simple to state -- perform some set of
functions on vector or array data. Current GPU programming techniques in
common languages for high performance computing (such as C, C++, and Fortran)
require this computational logic to be intermixed with the \emph{protocol
logic}, resulting in programs that are complex to write, maintain, and tune
for performance. Composition of independently developed program units that use
the GPU is similarly complex due to limited (or nonexistent) methods for
reasoning about the result of this composition in terms of lower level memory
usage and task creation.

This pattern suggests that programmers should consider abstracting the
interface to the GPU that constitutes the protocol logic of the program, so
that they can focus on the domain oriented computational logic of the program.
This will allow them to avoid obfuscating their computational logic with the
tedious details of interacting with the device. There are numerous ways to
approach this problem.

First, programmers could build a library of functions or objects which hide as
many details as possible. In fact this has already been done to a large
extent, with frameworks such as CUDA or OpenCL, and these libraries are the
primary way in which programmers already program for GPUs. This approach
mitigates many difficult issues, but provides only a minimal abstraction --
the library abstraction cannot hide the need to coordinate GPU activities such
as set up and memory management. Programmers might try to create their own
libraries on top of CUDA or OpenCL, but in most languages the object or
library facilities will not allow them to reason at a high semantic level. In
other words, they might be able to simplify some operations by specializing
them (e.g. ignore the possibility for multiple devices), but they cannot hide
them altogether.

Another approach is to design a domain-specific language for GPUs. Examples
include PyCUDA and OpenMPC. Domain-specific languages are typically not
customizable for higher-level application logic.

A technique that is becoming popular and is based on the OpenMP model of
directive-based code annotations hides much of the protocol logic by leaving
it to a compiler and runtime system to implement it. The PGI
Accelerate~\cite{pgi-accelerate} model and the HMPP programming
system~\cite{hmpp} are examples of this in compiler products. This approach
will, like OpenMP, make significant progress on lowering the barrier of entry
for programmers. Unfortunately, these methods provide too much a black-box to
the programmer, limiting their ability to customize the underlying GPU
protocol logic.

\subsection{The Twig Approach}

% What is the twig approach?  Give a little info here.

A key milestone in the history of programming languages was the introduction
of high level types into early languages. The ability of programmers to
separate the representation of a value of different types (e.g., floating
point versus integer; record types; arrays) allowed program code to be written
in terms of the mathematical abstractions that bits in memory represented
without exposing how the values were laid out in memory. We believe that a
similar type-level abstraction is possible in modern hybrid computer
architectures. The abstraction that must be moved to the type system is that
of the location of data within the system.

Changes in representation of basic values, such as integers or floats, is
often as simple as a type casting operator or an implicit type conversion as
allowed by the language. If residence information about data is part of the
type system, then movement of data within the memory hierarchy of a hybrid
machine can also be made as simple to express as traditional type coercions.
For the domain of GPGPU programming, Twig allows a domain specific type system
to be created in which this additional location information can be added to
the type of variables with corresponding type coercion operators that map to
lower level memory movement logic that defines the protocol between the host
and GPU device.

In addition, by abstracting the protocol operations away to high level type
coercion operators, Twig can support automated reasoning about programs that
allows optimizations to be performed related to the underlying protocol logic.
For example, if a programmer defines a sequence of statements that, based on
their type, state that data will move back and forth between the host and
device without modification on the host side, unnecessary copies of data can
be removed. This will increase program performance by reducing unnecessary
burden on the memory subsystem.

Current systems like CUDA and OpenCL allow the programmer to explicitly
allocate memory on one of the many devices that control some portion of the
memory within a machine, but require copies to be explicitly implemented by
the programmer. The result is fine grained control of the machine at the cost
of limited analysis and optimization of data movement within the memory
hierarchy. This makes writing complex programs difficult, particularly with
respect to performance. We believe that a Twig-based DSL for GPU programming
based on extending the type system of existing languages is a step towards
removing this burden from the programmer.

%!TEX root = twig-gpu.tex

\section{Related Work}

Numerous systems have been created in recent years to address the GPU
programming problem that provide an abstraction above low level interfaces such
as OpenCL or CUDA. These include the PGI Accelerate model~\cite{pgi-accelerate}
or the HMPP programming system~\cite{hmpp}. While both of these systems provide
an abstraction above the low level programming library, we believe that they
hide too much from the programmer. There is little room for tuning of the way
the lower level interface to the accelerator is used -- the programmer is
reliant on the tool vendor to provide a sufficiently tunable abstraction such
that working with this low level interface is unnecessary.

Unfortunately, in large applications, it is infeasible to assume that all
developers of the components that form the overall application will use the same
higher level abstraction method. This makes it challenging not only to tune the
code that bridges between devices, but to reason about how the code resulting
from the independent programming systems interacts. We address this by adopting
a code generation approach in which a single, low level target is used (such as
OpenCL). This approach addresses both the composability problem (all Twig code
maps to a single ``lingua franca'' for programming the hybrid system), and
exposes the implementation in the generated code to allow tuning and
modification by the end user.

The closest work to that which we describe is Reppy's Application Specific
Foreign Function Interface Generator, FIG~\cite{reppy06fig}. In that work, a
similar formal approach was taken specifically to the generate bindings between
programs in two different programming languages. We have found that very similar
issues arise in building bindings between devices in a hybrid system. These
overlaps include memory ownership and management, data marshalling, and managing
the flow of program control across the language or device boundary. Our work
builds upon that of Reppy and Song, and aims to provide a general-purpose tool
that is not tied to the Moby programming language. Twig's approach largely
subsumes the case of foreign function interface generation, and could
incorporate other interesting applications like mapping between two different
libraries written in the same language. This second case is of particular
interest when dealing with the problem of composing complex software from
smaller program units, where developers working independently may have chosen
different representations for data types that are semantically identical.

Kennedy's \emph{telescoping languages} work is related to ours in that it seeks
to support high-level programming by deferring challenging problems to automated
tools. In particular, the telescoping languages effort sought to provide the
ability to build high-level problem-solving languages that used scripting
languages to coordinate functionality present in domain-specific
libraries~\cite{kennedy00telescoping}. Much of the work then focused on compiler
optimization methods and targeting potentially distributed, grid-based
environments. Interestingly, one of the compilation techniques that was called
out as part of the telescoping languages strategy was that of automated
recognition and exploitation of identities. This would allow a compilation tool
to recognize instances when compositions of functions could be replaced with
more efficient equivalent implementations. As is described in this paper, we
adopt a similar strategy in identifying compositions of rules that are
equivalent to the identity function and can therefore be eliminated.

Code generation approaches have had notable success in the computational science
field, an exemplar being the Tensor Contraction Engine
(TCE)~\cite{baumgartner05synthesis}. The TCE allows computational chemists to
write tensor contraction operations in a high level language similar to
Mathematica, leaving it up to the TCE tool to generate the corresponding
collections of loops that implement the operations. The advantage of this
approach is that tedious and often error-prone nested loops over many large
arrays with a correspondingly large number of indices can be both machine
generated and optimized. Optimizations such as loop fusion, memory locality
management, and data distribution and partitioning in a parallel machine can all
be automated, versus previous approaches that required very labor intensive hand
written code. The TCE is not a general tool, and is only of use to programmers
working with similar tensor-based computations.

%!TEX root = twig-gpu.tex

\section{Design of the GPU DSL}

Details on the GPU DSL. This should be used to set up the next section so that
the reason Twig is talked about is clear.

%!TEX root = twig-language.tex

\section{Evaluation}

Evaluation. 

\begin{verbatim}
%typemap(in) int {
    $1 = PyInt_AsLong($input);
}
%typemap(out) int {
    $result = PyInt_FromLong($1);
}
\end{verbatim}

\begin{verbatim}
%typemap(in) int {
    $1 = PyInt_AsLong($input);
}
%typemap(out) int {
    $result = PyInt_FromLong($1);
}
\end{verbatim}

% Show a simple example written in both Swig and Twig, and show how Twig's composition operators make the code cleaner.

%!TEX root = twig-gpu.tex

\section{Conclusion}
\label{sec:conclusion}

We have introduced the concept of separating the protocol logic that is inherent to hybrid systems from the computational logic that forms the domain specific intent of a program that uses the system. We have demonstrated that a type-based approach can enforce this separation by making explicit in data types information related to both the locale in which data resides, and the representation of the data itself. By doing so, we allow the protocol logic of a program to be expressed via operations exclusively on located types. Many explicit programming chores become implicit features of the generated code, such as declaring intermediate values or reducing redundant memory movement. Finally, by adopting a code generation approach, we show that users of these higher level abstractions are not prohibited from both tuning the resultant code and composing together independently developed programs that utilize standardized hybrid programming libraries like OpenCL or CUDA.


This work was supported in part by the Department of Energy Office of Science,
Advanced Scientific Computing Research.

% \pagebreak
\bibliographystyle{abbrv}
\bibliography{twig-gpu}

\end{document}
