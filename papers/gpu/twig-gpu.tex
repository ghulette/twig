\documentclass[11pt]{article}
\usepackage{fullpage}
\usepackage{amssymb}
\usepackage{url}
\usepackage{parskip}

\title{Twig for GPUs}
\author{Geoffrey C. Hulette}

\begin{document}
\maketitle
\thispagestyle{empty}

\section{Introduction}

Programmers are often constrained by engineering requirements to produce work
in a mainstream language such as C. At the same time, there is an increasing
awareness of the benefits of \emph{domain-specific} programming languages
(DSLs). The goal of DSLs is to closely model the problem domain, which has at
least two benefits. First, programs that would be complicated or convoluted in
a traditional language may be expressed clearly and succinctly. Second, DSLs
may admit automated reasoning about programs, e.g. optimization for efficiency
in running time or space requirements. Where a DSL provides these benefits and
also generates C (or some other mainstream language) as output, it may be
useful and relatively easy to integrate with existing engineering processes.

The power of DSLs comes at a cost, however. DSLs typically have a unique
syntax and semantics that may be known primarily to the language's author. The
interpreter and other tools may be difficult to support or extend if the
original author becomes unavailable. It can also be difficult to extend DSLs
to support new constructs or rules. 

We introduce Twig, a DSL intended for ``ad-hoc'' modeling of a variety of
high-level program semantics, and admitting simple reasoning via equivalence
rules. The evaluation of a Twig program generates code in C (and potentially
other languages), lending it to integration in software existing projects. By
``ad-hoc'' we mean that Twig can be configured rather easily by application
programmers to serve as a DSL for different purposes, even within the same
application. This may alleviate some of the resistance to the adoption of
Twig-based DSLs in projects.

In this paper, we demonstrate how Twig can be used to describe some simple
kinds of GPU programming. GPU programming is a good example of the kinds of
problems we anticipate Twig being useful for -- it has a fairly simple
high-level structure (described below) but is tedious and complicated to write
by hand.

\section{Example Problem: GPGPU Programming}

Programming for GPUs can be a challenging task. There is a considerable amount
of protocol that must be invoked in order to set up and release the device,
move data to and from the device, synchronize processing at barriers. Yet for
many problems, the logic is fairly simple -- perform some function on a vector
or array.

This pattern suggests that programmers should consider abstracting the
interface to the GPU, so that they can focus on domain logic and ignore the
tedious details of interacting with the device. There are numerous ways to
approach this problem.

First, programmers could build a library of functions or objects which hide as
many details as possible. In fact this has already been done to a large
extent, with frameworks such as CUDA or OpenCL, and these libraries are the
primary way in which programmers already program for GPUs. This approach
mitigates many difficult issues, but provides only a minimal abstraction --
the library abstraction cannot hide the need to coordinate GPU activities such
as set up and memory management. Programmers might try to create their own
libraries on top of CUDA or OpenCL, but in most languages the object or
library facilities will not allow them to reason at a high semantic level. In
other words, they might be able to simplify some operations by specializing
them (e.g. ignore the possibility for multiple devices), but they cannot hide
them altogether.

Another approach is to design a domain-specific language for GPUs. Examples
include PyCUDA and OpenMPC. Domain-specific languages are typically not
customizable for higher-level application logic.

\section{Related Work}

Discuss PyCUDA and OpenMPC. 

Yampa?

Other general-purpose DSLs?

\section{Twig}

Overview of how Twig works, basic semantics, code generation, simple example.

Focus on importance of generating C - integration with existing methods, type-directed generation.

\subsection{Reductions}

Explain reductions and how they can be used in domain- or application-specific
ways.

Explain built-in reductions. Maybe discuss idea of a normal form, although
this might be better in a later paper.

\section{Implementation}

Talk briefly about how Twig is implemented, and how to use it to generate
code.

\section{Evaluation}

Show some GPU algorithm encoded in Twig. Discuss advantages over simple APIs
(restricted flexibility equals improved ability to reason).

Show how reductions eliminate redundant copies.

Compare performance versus naive generation.

Also compare against PyCUDA or another system?

\section{Future work}

Future work.

\section{Conclusion}


% \section{Section Title}
% \input{section_file}

% \pagebreak
% \bibliographystyle{plain}
% \bibliography{references}

\end{document}
