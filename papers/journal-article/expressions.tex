%!TEX root = twig.tex

\subsection{Expressions}
\label{sec:expressions}

Twig \emph{expressions} can be either \emph{primitive rules}
(Section~\ref{sec:semantics:prims}) or else built from other
expressions using operators (Section~\ref{sec:semantics:ops}). We
denote the set of all expressions $S$. An expression $s \in S$
maps ground terms $T$ to elements of the set $(T \times M) \cup
\{\bot\}$, i.e., either a pair $(t',m)$ where $t' \in T$ and $m
\in M$ is a \emph{block} of generated code in the set $M$ (see
Chapter~\ref{ch:code-gen}), or else the special, distinguished
value $\bot$. Formally, $s \in S$ is a function:

\[
s : T \to ((T \times M) \cup \{\bot\})
\]

Following Fig's notation, we use $\bot$ to denote ``failure.'' In
particular, $\bot$ is used in the semantics for the operators
described in Section~\ref{sec:semantics:ops}.

As with System S and Fig, Twig allows expressions to be named. An
expression's name may be used in place of itself within other
expressions. The syntax is

\[
v \mtt{ = } s
\]

where $v$ is an expression identifier and $s \in S$ is an
expression of the form described below. A Twig program is a list
of such name/expression assignments. To prevent circular
references, expressions may only reference names that have been
previous defined (i.e., appearing before the expression in the
program text). For the same reason, expressions may not reference
their own name. For example:

\begin{verbatim}
foo = foo ; bar
\end{verbatim}

is not allowed, because the definition of \texttt{foo} references
\texttt{foo} itself. Programs requiring recursive expressions
should use the \texttt{\#fix} fixed-point operator instead.

There is a special expression name, \texttt{main}, which
designates the top-level expression for the program.
