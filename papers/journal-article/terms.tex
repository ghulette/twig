%!TEX root = twig.tex

\subsection{Terms}
\label{sec:terms}

Twig programs operate on values called \emph{terms}. Terms
represent tree-structured data with labeled internal nodes, a
versatile data type useful in a variety of applications. For
example, abstract syntax trees are naturally represented as terms.

We define the grammar for constructing terms as follows:

\[
t \;\mbox{:=}\; x \;|\; c \;|\; f(t_1,\ldots,t_n)
\]

A term is either a variable $x$, a constant $c$, or the
application of a \emph{constructor} $f$ to at least one other
term. In Twig's syntax, constants and constructors can be any
string of characters beginning with a lower-case letter. The only
string excluded from this condition is the special constructor
\texttt{tuple} (see Section~\ref{sec:tuples}, below). Variables
are represented by strings of characters beginning with a capital
letter; in our presentation we will typically use a single capital
letter only, e.g., \texttt{X} or \texttt{Y}. We use a fixed-width
font when presenting terms, e.g., the constant term
\texttt{myterm}, or the constructed term with a variable
\texttt{cons(myterm1,X)}.

Following the notation of System S, we denote the set of all
variables $\mathcal{X}$, and the set of all terms containing
variables $T(\mathcal{X})$. We denote the set of all terms without
variables, known as \emph{ground terms}, as $T$.

In many systems, terms are typed through the use of
\emph{signatures}~\cite{baader98rewriting}. That is, certain terms
can be defined as valid in a particular domain, and terms not
found to conform can be flagged as errors. Twig could be extended
to support term signatures, and indeed this might be useful. In
our current work, however, we consider only untyped terms.

For the purposes of Twig's semantics, the meaning of a particular
term (other than \texttt{tuple}s) is abstract. Terms are defined
by their use in the program's \emph{rules}, described below.


\subsubsection{Tuples}
\label{sec:tuples}

Twig recognizes a special kind of term: tuples. The tuple elements
are represented as the sub-terms of a term with a special
constructor: \texttt{tuple}. Tuples may have any length. Twig's
syntax equates the absence of any constructor with the presence of
the \texttt{tuple} constructor. For example, the syntax
\texttt{(string,int)} is interpreted as the
\texttt{tuple(string,int)}. This term represents a tuple of length
two, whose first element is \texttt{string} and whose second
element is \texttt{int}.

The \emph{size} of a tuple is simply the cardinality of its
children. We will sometimes write $\mathtt{tuple}_n(\ldots)$ to
indicate a tuple of length $n$, where the length is not otherwise
clear from the context.

One complication arises since we permit tuples to be nested to
arbitrary depth. For example the term

\[
\mathtt{tuple(tuple(int,float),tuple(double))} 
\]

is a nested tuple. In our semantics, we will require the
\emph{width} of a tuple, defined as

\[
\mbox{width}(t) = \left\{
  \begin{array}{cl}
    \sum^{i=1}_{n} \mbox{width}(t_i) 
      & \mbox{if } t = \mathtt{tuple}(t_1,\ldots,t_n)\vspace{2mm}\\
    1 & \mbox{otherwise}
  \end{array}
\right.
\]

Intuitively, the width of a tuple corresponds to its size after
being ``flattened,'' where the elements of nested tuples are
pushed up, recursively, to the top level. If we flattened the
tuple in the example above, we would get

\[
\mathtt{tuple(int,float,double)}
\]

and its width would be three.


\subsubsection{Terms representing types}

In Twig, terms are used to represent types in a target language.
For example, we use terms such \texttt{int} and \texttt{float} to
represent primitive types in C. Terms built with constructors can
represent types with some structure, e.g., the term
\texttt{ptr(int)} can represent a C pointer to an integer. More
complicated terms may involve multiple children, and may be nested
to any depth. For example, the term

\[
\mathtt{struct(int,float,struct(ptr(char)))}
\]

can represent a structure with three fields: an \texttt{int}, a
\texttt{float}, and a second structure with a single string
(pointer to \texttt{char}) field.

The mapping between terms and types in the target language is a
configuration option, customizable for a particular domain. The
mapping need not be injective, that is, multiple terms in Twig may
represent a single type in the target language. For example, you
might have the distinct terms \texttt{string} and
\texttt{ptr(char)} both map to a \texttt{char} pointer in C.
