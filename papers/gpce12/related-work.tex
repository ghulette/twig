%!TEX root = twig-language.tex

\section{Related Work}

There are many tools which incorporate a notion of typemaps. The
idea originated in SWIG~\cite{swig}, a tool used for generating
foreign function interfaces from C header files. Typemaps in Swig
are robust, and support user customization. However, the semantics
of Swig's typemaps are ad-hoc, inflexible, and specialized to
generate C code.

FIG~\cite{fig} introduced the notion of application-specific
typemaps, and is similar in spirit to our own work. Unlike Twig,
FIG is specialized to generate code for Moby~\cite{moby-classes}.
Moby is a convenient target language -- its declarative structure
and semantics are amenable to generation via System
S~\cite{fisher00interop}. Indeed, FIG takes advantage of this fact
by providing rules specific to Moby. Moby is not nearly as
ubiquitous as C, however, and therefore not a very practical
target language for many people.

There are other tools which utilize typemaps, particularly
foreign-function interface generators such as
Charon~\cite{moby-interop-framework} or NLFFIGen~\cite{blume01}.
Twig might complement these systems well, providing a foundational
semantics for their typemaps along with the ability
to generate code for a variety of target languages.

Our abstract code generation model is based in part on our own
previous work on the Wool~\cite{wool} language for workflow
programming.
