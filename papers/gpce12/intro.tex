%!TEX root = twig-language.tex

\section{Introduction}

Twig is a language for writing \emph{typemaps} -- programs that
transform data from one type to another while preserving, as much
as possible, the underlying meaning of the data. Typemaps have
proven useful in many kinds of programming and especially
automated code generation. The best-known application of typemaps
has been for multi-language programming. In this domain, for
example, a programmer may wish to pass an integer from a Python
program across a foreign function interface to a C function, where
a C \texttt{int} is expected. A typemap can be used to describe
the generic transformation from Python integers to C integers,
enabling a tool such as SWIG to generate the conversion code
automatically. The C function is exposed to Python via the
generated wrapper.

There are a number of existing languages for typemaps and tools
which generate code from them. Twig builds on existing typemap
tools in several ways.

First, Twig's typemaps are composable, i.e., complex typemap
transformations may be constructed by combining simpler ones. Our
typemap semantics are based on those found in Fig\cite{fig} and
System S\cite{system-s}, but we extend and refine those systems.

Second, Twig incorporates a robust, formal model of code
generation in its semantics. This allows Twig to generate code
based on typemaps for different target languages.

Finally, Twig includes a facility for \emph{reducing} typemaps by
exploiting identity relationships among typemap expressions. Some
reductions are based on a formal algebra of typemaps, while others
are domain-specific and provided by the user. In forthcoming work
we show how user-supplied typemap reductions can be used to
optimize certain transformations. We will not cover reductions
further in this paper.

In this paper, we will describe Twig's formal language structure,
and then show how this structure allows us to concisely express
complex typemaps. First, we review existing approaches to typemaps
and related problems. Second, we will present Twig's semantics.
Third, we walk through a typemap example in SWIG, and show how the
typemaps can be expressed more concisely in Twig. We conclude with
ideas for future work.
